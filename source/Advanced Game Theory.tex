\documentclass[12pt]{extreport} % Schriftgröße: 8pt, 9pt, 10pt, 11pt, 12pt, 14pt, 17pt oder 20pt

% Language Setup (Deutsch)
\usepackage[utf8]{inputenc} 
\usepackage[english]{babel}

%% Packages
\usepackage{scrextend}
\usepackage{amssymb}
\usepackage{amsthm}
\usepackage{amsmath}
\usepackage[inline]{enumitem}
\usepackage{changes}
\usepackage{chngcntr}
\usepackage{cmap}
\usepackage{color}
\usepackage{csquotes}
\usepackage{float}
\usepackage{hyperref}
\usepackage{footnote}

\usepackage{lmodern}
\usepackage{makeidx}
\usepackage{mathtools} 
\usepackage{xpatch}
\usepackage{pgfplots}
\usepackage{stmaryrd}
\usepackage{pbox}
\usepackage{apptools}
\usepackage{booktabs}
\usepackage{dsfont}
\usepackage{graphicx}
\usepackage{mathrsfs}
\usepackage[square,numbers]{natbib}
\usepackage{nicefrac}
\usepackage{pgf}
\usepackage{pgfplots}
\usepackage{tikz}
\usepackage{tocloft}
\usepackage{url}
\usepackage{xpatch}
\usepackage{microtype}
\usepackage{pgfplots}
\usepackage{minibox}
\usepackage{xcolor}
\usepackage{sgame} % Game theory packages
\usepackage{subfig} % Manipulation and reference of small or sub figures and tables

\makesavenoteenv{tabular}
\usepgfplotslibrary{fillbetween}
\usetikzlibrary{patterns}
\usetikzlibrary{decorations.markings}
\usetikzlibrary{calc, intersections}
\usetikzlibrary{trees, calc} % For extensive form games
\pgfplotsset{compat=1.7}
\usetikzlibrary{calc}	
\usetikzlibrary{matrix}	

\usepgfplotslibrary{fillbetween}
\usetikzlibrary{patterns}
\usetikzlibrary{decorations.markings}
\usetikzlibrary{calc, intersections}
\usetikzlibrary{trees, calc} % For extensive form games
\pgfplotsset{compat=1.7}
\usetikzlibrary{calc}	
\usetikzlibrary{matrix}	

% Options
\makeatletter%%  
  % Linkfarbe, {0,0.35,0.35} für Türkis, {0,0,0} für Schwarz 
  \definecolor{linkcolor}{rgb}{0,0.35,0.35}
  % Zeilenabstand für bessere Leserlichkeit
  \def\mystretch{1.5} 
  % Publisher definieren
  \newcommand\publishers[1]{\newcommand\@publishers{#1}} 
  % Enumerate im 1. Level: \alph für a), b), ...
  \renewcommand{\labelenumi}{\alph{enumi})} 
  % Enumerate im 2. Level: \roman für (i), (ii), ...
  \renewcommand{\labelenumii}{(\roman{enumii})}
  % Zeileneinrückung am Anfang des Absatzes
  \setlength{\parindent}{0pt} 
  % Verweise auf Enumerate, z.B.: 3.2 a)
  \setlist[enumerate,1]{ref={\thesatz ~ \alph*)}}
  % Für das Proof-Environment: 'Beweis:' anstatt 'Beweis.'
  \xpatchcmd{\proof}{\@addpunct{.}}{\@addpunct{:}}{}{} 
  % Nummerierung der Bilder, z.B.: Abbildung 4.1
  \@ifundefined{thechapter}{}{\def\thefigure{\thechapter.\arabic{figure}}} 
\makeatother%

% Meta Setup (Für Titelblatt und Metadaten im PDF)
\title{Advanced Game Thoerie}
\author{Prof. Dr. Clemens Puppe}
\date{(inoffizielles Skript) ~\vspace{0.2cm} \\ Wintersemester 2016/17}
\publishers{Karlsruher Institut für Technologie}

%% Math. Definitions
\newcommand{\C}{\mathbb{C}}
\newcommand{\N}{\mathbb{N}}
\newcommand{\Q}{\mathbb{Q}}
\newcommand{\R}{\mathbb{R}}
\newcommand{\Z}{\mathbb{Z}}

%% Theorems (unnamedtheorem = Theorem ohne Namen)
\newtheoremstyle{named}{}{}{\normalfont}{}{\bfseries}{:}{0.25em}{#2 \thmnote{#3}}
\newtheoremstyle{itshape}{}{}{\itshape}{}{\bfseries}{:}{ }{}
\newtheoremstyle{normal}{}{}{\normalfont}{}{\bfseries}{:}{ }{}
\renewcommand*{\qed}{\hfill\ensuremath{\square}}

\theoremstyle{named}
\newtheorem{unnamedtheorem}{Theorem} \counterwithin{unnamedtheorem}{chapter}
\newtheorem*{unnamedtheorem*}{Theorem} 

\theoremstyle{itshape}
\newtheorem{theorem}[unnamedtheorem]{Theorem}	
\newtheorem{satz}[unnamedtheorem]{Satz}	
\newtheorem*{definition}{Definition}

\theoremstyle{normal}
\newtheorem{beispiel}[unnamedtheorem]{Beispiel}
\newtheorem{example}[unnamedtheorem]{Example}
\newtheorem{folgerung}[unnamedtheorem]{Folgerung}
\newtheorem{hilfssatz}[unnamedtheorem]{Hilfssatz}
\newtheorem{proposition}[unnamedtheorem]{Proposition}
\newtheorem{anwendung}[unnamedtheorem]{Anwendung}
\newtheorem{anwendungen}[unnamedtheorem]{Anwendungen}
\newtheorem*{beispiel*}{Beispiel}
\newtheorem*{example*}{Example}
\newtheorem*{beispiele}{Beispiele}
\newtheorem*{examples}{Examples}
\newtheorem*{bemerkung}{Bemerkung} 
\newtheorem*{comment*}{Comment} 
\newtheorem*{bemerkungen}{Bemerkungen}
\newtheorem*{bezeichnung}{Bezeichnung}
\newtheorem*{eigenschaften}{Eigenschaften}
\newtheorem*{erinnerung}{Erinnerung}
\newtheorem*{folgerung*}{Folgerung}
\newtheorem*{folgerungen}{Folgerungen}
\newtheorem*{hilfssatz*}{Hilfssatz}
\newtheorem*{regeln}{Regeln}
\newtheorem*{schreibweise}{Schreibweise}
\newtheorem*{schreibweisen}{Schreibweisen}
\newtheorem*{uebung}{Übung}
\newtheorem*{vereinbarung}{Vereinbarung}

%% Template
\makeatletter%
\DeclareUnicodeCharacter{00A0}{ } \pgfplotsset{compat=1.7} \hypersetup{colorlinks,breaklinks, urlcolor=linkcolor, linkcolor=linkcolor, pdftitle=\@title, pdfauthor=\@author, pdfsubject=\@title, pdfcreator=\@publishers}\DeclareOption*{\PassOptionsToClass{\CurrentOption}{report}} \ProcessOptions \def\baselinestretch{\mystretch} \setlength{\oddsidemargin}{0.125in} \setlength{\evensidemargin}{0.125in} \setlength{\topmargin}{0.5in} \setlength{\textwidth}{6.25in} \setlength{\textheight}{8in} \addtolength{\topmargin}{-\headheight} \addtolength{\topmargin}{-\headsep} \def\pulldownheader{ \addtolength{\topmargin}{\headheight} \addtolength{\topmargin}{\headsep} \addtolength{\textheight}{-\headheight} \addtolength{\textheight}{-\headsep} } \def\pullupfooter{ \addtolength{\textheight}{-\footskip} } \def\ps@headings{\let\@mkboth\markboth \def\@oddfoot{} \def\@evenfoot{} \def\@oddhead{\hbox {}\sl \rightmark \hfil \rm\thepage} \def\chaptermark##1{\markright {\uppercase{\ifnum \c@secnumdepth >\m@ne \@chapapp\ \thechapter. \ \fi ##1}}} \pulldownheader } \def\ps@myheadings{\let\@mkboth\@gobbletwo \def\@oddfoot{} \def\@evenfoot{} \def\sectionmark##1{} \def\subsectionmark##1{}  \def\@evenhead{\rm \thepage\hfil\sl\leftmark\hbox {}} \def\@oddhead{\hbox{}\sl\rightmark \hfil \rm\thepage} \pulldownheader }	\def\chapter{\cleardoublepage  \thispagestyle{plain} \global\@topnum\z@ \@afterindentfalse \secdef\@chapter\@schapter} \def\@makeschapterhead#1{ {\parindent \z@ \raggedright \normalfont \interlinepenalty\@M \Huge \bfseries  #1\par\nobreak \vskip 40\p@ }} \newcommand{\indexsection}{chapter} \patchcmd{\@makechapterhead}{\vspace*{50\p@}}{}{}{}
	% Titlepage
	\def\maketitle{ \begin{titlepage} 
			~\vspace{3cm} 
		\begin{center} {\Huge \@title} \end{center} 
	 		\vspace*{1cm} 
	 	\begin{center} {\large \@author} \end{center} 
	 	\begin{center} \@date \end{center} 
	 		\vspace*{7cm} 
	 	\begin{center} \@publishers \end{center} 
	 		\vfill 
	\end{titlepage} }
\makeatother%

% Indexdatei erstellen
\makeindex 

\begin{document}

\pagenumbering{Alph}
\begin{titlepage}
	\maketitle
	\thispagestyle{empty}
\end{titlepage}
\pagenumbering{arabic}
	
% Inhaltsverzeichnis
\tableofcontents
\thispagestyle{empty} 
  
% Skript - Anfang 
\chapter{Noncooperative Games}

\section{Basic Elements of Noncooperative Games}


\begin{definition}
	A \textbf{game} is a formal representation of a situation in which a number of individuals interact in a setting of strategic interdependence.
	
	\begin{itemize}
		\item The players: Who is involved?
		\item The rules: Who moves when? What do they know when they move? What can they do?
		\item The outcomes: For each possible set of actions by the players, what is the outcome of the game?
		\item The payoffs: What are the players' preferences over the possible outcomes?
	\end{itemize} 
\end{definition} 


\begin{example}[of simultaneous move games]  ~\
	\begin{enumerate}
		\item Matching Pennies
			\begin{figure}[h!] \centering
  				\begin{game}{2}{2}[Player $1$][Player $2$]
   	    			   	 	&	  Heads    &  Tails   \\
   	 				Heads   &    $-1, 1$   & $1, -1$  \\
   	 				Tails   &    $1, -1$   & $-1, 1$  \\
   				\end{game}
			\end{figure} 
		\item Meeting in New York ~\\
			\begin{figure}[h!] \centering
  				\begin{game}{2}{2}[Player $1$][Player $2$]
   	    			   	 		&	Empire State    &  Grand Central   \\
   	 				Empire State   &    $100, 100$   & $0, 0$  \\
   	 				Grand Central   &    $0,0$   & $100, 100$  \\
   				\end{game}
			\end{figure}
		\item Examples of (simple) dynamic games
			 \begin{figure}[h!]
 	\centering
	\caption*{Prisoner's Dilemma in Extensive-form \label{PDExtensive}}
    \begin{tikzpicture}[thin,
      level 1/.style={sibling distance=40mm},
      level 2/.style={sibling distance=25mm},
      level 3/.style={sibling distance=15mm},
      every circle node/.style={minimum size=1.5mm,inner sep=0mm}]
      
      \node[circle,draw,label=above:$P_1$] (root) {}
        child { node [circle,fill,label=above:$P_2$] {}
          child { 
            node {$($-$1,1)$}
            edge from parent
              node[left] {$H$}}
          child { 
            node {$(1,$-$1)$}
            edge from parent
              node[right] {$T$}}
          edge from parent
            node[left] {$H$}}
        child { node [circle,fill,label=above:$P_{2}$] {}
          child { 
            node {$(1,$-$1)$}
              edge from parent
                node[left] {$H$}}
          child { 
            node {$($-$1,1)$}
              edge from parent
                node[right] {$T$}}
           edge from parent
             node[right] {$T$}};
    \end{tikzpicture}
  \end{figure}
		\item Matching Pennies Version C
	% Node styles
	\tikzset{
	% Two node styles for game trees: solid and hollow
	solid node/.style={circle,draw,inner sep=1.5,fill=black},
	hollow node/.style={circle,draw,inner sep=1.5}
	} 

\begin{figure}[h!]
	\centering
	\begin{tikzpicture}[scale=1.5,font=\footnotesize]
	% Specify spacing for each level of the tree
	\tikzstyle{level 1}=[level distance=15mm,sibling distance=40mm]
	\tikzstyle{level 2}=[level distance=15mm,sibling distance=22mm]
	\tikzstyle{level 3}=[level distance=15mm,sibling distance=15mm]
	% The Tree
	\node(0)[hollow node,label=above:{$P1$}]{}
	child{node(1)[solid node]{}
		child{node[label=below:{$($-$1,1)$}]{} edge from parent node[left]{$H$}}
		child{node(3)[label=below:{$(1,$-$1)$}]{} edge from parent node[right]{$T$}}
		edge from parent node[left,xshift=-3]{$H$}
	}
	child{node(2)[solid node]{}
		child{node[label=below:{$(1,$-$1)$}]{} edge from parent node[left]{$H$}}
		child{node[label=below:{$($-$1,1)$}]{} edge from parent node[right]{$T$}}
		edge from parent node[right,xshift=3]{$T$}
	};
	% information set
	\draw[dashed,rounded corners=10]($(1) + (-.2,.25)$)rectangle($(2) +(.2,-.25)$);
	% specify mover at 2nd information set
	\node at ($(1)!.5!(2)$) {$P2$};
	\end{tikzpicture}
\end{figure}

	\end{enumerate}
\end{example} ~\

\begin{definition}[Information] ~\
	\begin{enumerate}
		\item \textbf{Information Set:} A player doesn't know which of the nodes in the information set she is actually at. Therefore, at any decision node in a player's information set, there must be the same possible actions.
		\item \textbf{Perfect Information:} A game is said to be of perfect information if each information set contains a single decision node. Otherwise, it is a game of \textbf{imperfect information}.
	\end{enumerate}
\end{definition}

\begin{definition}[Extensive Form Game] ~\
	A game in \textbf{extensive form} consists of:
	\begin{enumerate}[label=(\roman*\upshape)]
		\item A finite set of nodes $\mathcal{X}$, a finite set of possible actions $\mathcal{A}$, and a finite set of players $\{1, \dotsc, l\}$.
		\item A funktion $p \colon \mathcal{X} \rightarrow \{ \mathcal{X} \cup \emptyset \}$ specifying a single immediate predecessor of each node $x$; $p(x) \in \mathcal{X}$ expect for one element $x_{0}$, the \textbf{initial node}. The immediate \textbf{successor node} of $x$ are $s(x) = p^{-1}(x)$. ~\\
			To have a tree structure, a predecessor can never be a successor and vice versa. The set of \textbf{terminal nodes} $T = \{ x \in \mathcal{X} \colon s(x) = \emptyset \}$. All other nodes $X \setminus T$ are \textbf{decision nodes}.
		\item A function $\alpha \colon \mathcal{X} \setminus \{ x_{0} \} \rightarrow \mathcal{A}$ giving the action that leads to any non-initial node $x$ from its immediate predecessor $p(x)$ with $x', x'' \in s(x); x' \neq x'' \Rightarrow \alpha(x') \neq \alpha(x'')$. The set of choices at decision node $x$ is $c(x) = \{ a \in \mathcal{A} \colon a = \alpha(x') \text{ for some } x' \in s(x) \}$.
		\item A collection of information sets $\mathcal{H}$, and a function $H \colon \mathcal{X} \rightarrow \mathcal{H}$ assigning each decision node $x$ to an information set $H(x) \in \mathcal{H}$ with $c(x) = c(x')$ if H(x) = $H(x')$. ~\\
			The choices available at information set $H$ can be written as
			$$ C(H) = \{ a \in \mathcal{A} \colon a \in c(x) \text{ for } x \in H \}. $$
		\item A function $\iota \colon \mathcal{H} \rightarrow \{ 0, 1, \dotsc, l \}$ assigning a player to each information set ($i = 0$ 'nature'). ~\\
			The collection of player i's information set is denoted by
			$$ \mathcal{H}_i = \{ H \in \mathcal{H} \colon i = \iota(H) \}. $$
		\item A function $\rho \colon \mathcal{H}_0 \times \mathcal{A} \rightarrow [0,1]$ assigning a probability to each action of nature with $\rho(H,a) = 0$ if $a \notin C(H)$ und $\sum_{a \in C(H)} \rho(H, a) = 1$ for all $H \in \mathcal{H}_{0}$.
		\item A collection of payoff function $u = \{ u_1(\cdot), \dotsc, u_l(\cdot) \}$, where $u_i \colon T \rightarrow \R$.
	\end{enumerate}
	\textbf{A game in extensive form:} $\Gamma_E = \{ \mathcal{X}, \mathcal{A}, I, p(\cdot), \alpha(\cdot), \mathcal{H}, H(\cdot), \iota(\cdot), \rho(\cdot), u \}$.
\end{definition}

\begin{comment*}
	Restrictions of this definition:
	\begin{enumerate}
		\item Finite set of actions
		\item Finite number of moves
		\item Finite number 	of players
	\end{enumerate}
\end{comment*}

\begin{definition}[Strategy]
	Let $\mathcal{H}_i$ denote the collection of player $i$'s information sets, $\mathcal{A}$ the set of possible actions in the game, and $C(H) \subset \mathcal{A}$ the set of actions possible at information set $H$. A \textbf{strategy} for player $i$ is a function $s_i \colon \mathcal{H}_i \rightarrow \mathcal{A}$ such that $s_i(H) \in C(H)$ for all $H \in \mathcal{H}_i$.
\end{definition}

\begin{definition}[Normal Form Representation]
	For a game with $I$ players, the \textbf{normal form representation} $\Gamma_N$ specifies for each player $i$ a set of strategies $\mathcal{S}_{i}$ (with $s_i \in \mathcal{S}_i$) and a payoff function $u_i(s_1, \dotsc, s_l)$, formally 
	$$ \Gamma_N = [I, \{ S_i \}, \{ u_i(\cdot) \}]. $$
\end{definition} 

\newpage

\begin{definition} ~\
	\begin{enumerate}
		\item $s_i \colon \mathcal{H}_i \rightarrow \mathcal{A}$ describes deterministic choices at each $H \in \mathcal{H}_i$ and is called a \textbf{pure strategy}
		\item a \textbf{mixed strategy} is a probability distribution over all pure strategies $\sigma_i \colon \mathcal{S}_i \rightarrow [0, 1]$, with $\sigma_i(s_i) \geq 0$ and $\sum_{s_i \in \mathcal{S}_i} \sigma_i(s_i) = 1$.
		\item player $i$'s set of possible mixed strategies can be associated with the points of the simplex $\Delta(\mathcal{S}_i)$, called the \textbf{mixed extension} of $\mathcal{S}_i$.
		\item since we assume that individuals are expected utility maximisers, player $i$'s utility of a profile of mixed strategies $\sigma = \left( \sigma_i, \dotsc, \sigma_l \right)$ is given by
			$$ u_i(\sigma) = \sum_{s \in \mathcal{S}} [\sigma_1(s_1) \cdot \sigma_2(s_2) \cdot \dotsc \cdot \sigma_l(s_l)] \cdot u_i(s), $$
			where $s = (s_1, \dotsc, s_l)$.
	\end{enumerate}
\end{definition}

\begin{definition}[Behaviour Strategy]
	Given an extensive form game $\Gamma_E$, a \textbf{behaviour strategy} for player $i$ specifies for every information set $h \in \mathcal{H}_i$ and action $a \in C(H)$, a probability $\lambda_i(a, H) \geq 0$, with
	$$ \sum_{a \in C(H)} \lambda_i(a, H) = 1 \text{ for all } H \in \mathcal{H}_i. $$
\end{definition}

\begin{definition}[Perfect Recall]
	A player has \textbf{perfect recall} if he doesn't \enquote{forget} what she once knew, including her own actions.
\end{definition}

\begin{theorem}
	If $\Gamma_E$ is an extensive form game with perfect recall, then for any mixed strategy there is an outcome equivalent behaviour strategy and vice versa.	
\end{theorem}
 ~\newpage
\section{Rationalisable Strategies}

Central question of Game Theory: What should we expect to observe in a game played by rational players? Or more precisely: What should we expect to observe in a game played by rational players who are fully knowledgeable about the structure of the game and each others' rationality? ~\\

We first address the above question for simultaneous-move games, which we study using their normal form representation. We use the following notation:
\begin{itemize}
	\item $\Gamma_N = [I, \{ S_i \}, \{ u_i(\cdot)]$ if we consider pure strategies only, ~\\
		$\Gamma_{N} = [I, \{ \Delta(S_i)\}, \{ u_i(\cdot) \}]$ if we allow for mixed strategies
	\item $s_{-i} = (s_1, \dotsc, s_{i-1}, s_{i+1}, \dotsc, s_l) \in \mathcal{S}_{-i}$ where $\mathcal{S}_{-i} = S_1 \times \dotsc \times S_{i-1} \times S_{i+1} \times \cdots \times S_{l}$
	\item $s = (s_i, s_{-i})$
\end{itemize}

\begin{example}[Prisoners' Dilemma] ~\\
			\begin{figure}[h!] \centering
  				\begin{game}{2}{2}[Player $1$][Player $2$]
   	    			   	 	&	  don't confess    &  confess   \\
   	 				don't confess   &    $-2, -2$   & $-10, -1$  \\
   	 				confess   &    $-1, -10$   & $-5, -5$  \\
   				\end{game}
			\end{figure}
\end{example}

What should we expect to observe in the Prisoners' Dilemma?

\begin{definition}[Strictly Dominant Strategy]
	A strategy $s_i \in \mathcal{S}_i$ is \textbf{strictly dominant} for player $i$ in game $\Gamma_N = ]I, \{  \mathcal{S}_i \}, \{ u_i(\cdot)\}]$ if for all $s_i' \neq s_i$:
	$$ u_{i}(s_i, s_{-i}) > u_i(s_i', s_{-i})  $$
	for all $s_{i} \in \mathcal{S}_{-i}$.
\end{definition}
Applied to Prisoner's Dilemma: Confess is a strictly dominant strategy for each player.

\begin{definition}[Strictly Dominated Strategy]
	$s_i \in \mathcal{S}_i$ is \textbf{strictly dominated} for player $i$ in game $\Gamma_N$ if there exists another strategy $s_i' \in \mathcal{S}_i$ such that:
	$$ u_i(s_i', s_{-i}) \geq u_i(s_i, s_{-i}) $$
	for all $s_{-i} \in \mathcal{S}_{-i}$. In this case we say that $s_i'$ strictly dominates $s_i$.
\end{definition}


\begin{definition}[Weakly Dominated Strategy]
	$s_i \mathcal{S}_{i}$ is \textbf{weakly dominated} for player $i$ in game $\Gamma_N$ if there exists another strategy $s_i' \in \mathcal{S}_i$ such that:
	$$ u_i(s_i', s_{-i}) \geq u_i(s_i, s_{-i}) $$
	for all $s_{-i} \in \mathcal{S}_{-i}$, with strict inequality for at least one $s_{-i}$.
\end{definition}

\begin{example} ~\\
	\begin{figure}[h!] \centering
  				\begin{game}{3}{2}[Player $1$][Player $2$]
   	    			   	 	&	  L    &  R   \\
   	 				U   &    $1, -1$   & $-1, 1$  \\
   	 				M   &    $-1, 1$   & $1, -1$  \\
   					D   &    $-2, 5$   & $-3, 2$  \\
   				\end{game} $ \Rightarrow D \text{ is strictly dominated by } U \text{ and } M.$
   					\begin{game}{3}{2}[Player $1$][Player $2$]
   	    			   	 	&	  L    &  R   \\
   	 				U   &    $5, 1$   & $4, 0$  \\
   	 				M   &    $6, 0$   & $3, 1$  \\
   					D   &    $6,4$   & $4, 4$  \\
   				\end{game} $\Rightarrow U \text{ and } M \text{ are weakly dominated by } D.$
	\end{figure}
\end{example} ~\\


\begin{example}[Prisoners’ Dilemma – A Variation]
	Assume Prisoner 1 is the district attorney’s brother: If neither player confesses, player 1 is free	 ~\\
	\begin{figure}[h!] \centering
  				\begin{game}{2}{2}[Player $1$][Player $2$]
   	    			   	 	&	  don't confess    &  confess   \\
   	 				don't confess   &    $0, -2$   & $-10, -1$  \\
   	 				confess   &    $-1, -10$   & $-5, -5$  
   	   				\end{game} $$\Rightarrow D \text{ is strictly dominated by } U \text{ and } M.$$
	\end{figure} ~\
	
	$\Rightarrow \text{ Player 1 has no dominant strategy anymore}$.
\end{example}

In this game, the iterated elimination of strictly dominated strategies still leads to a unique prediction. In general, the order of elimination of strictly dominated strategies does not matter! How about iterated elimination of weakly dominated strategies?

\begin{definition}
	A strategy $\sigma_i \in \Delta(\mathcal{S}_i)$ is strictly dominated for $i$ in game $\Gamma_{N} = [I, \{ \Delta(\mathcal{S}_i)\}, \{ u_i(\cdot) \}]$ if there exists another strategy $\sigma_i' \in \Delta(\mathcal{S}_i)$ such that for all $\sigma_{-i} \in \Pi_{j \neq i} \Delta(\mathcal{S}_{j})$:
	$$ u_{i}(\sigma_i', \sigma_{-i}) > u_i(\sigma_i, \sigma_{-i}). $$
\end{definition}


\begin{proposition}
	Player $i$'s pure strategy $s_i \in \mathcal{S}_i$ is strictly dominated in a game $\Gamma_N = [I, \{ \Delta(\mathcal{S}_i)\}, \{ u_i(\cdot)\}]$ if and only if there exists another strategy $\sigma_i' \in \Delta(\mathcal{S}_i)$ such that
	$$ u_i(\sigma_i', s_{-i}) > u_i(s_i, s_{-i}) \text{ for all } s_{-i} \in \mathcal{S}_{-i}. $$
	
	\begin{proof}
		This follows because we can write
		$$ u_i(\sigma_i', \sigma_{-i}) - u_i(s_i, \sigma_{-i}) = \sum_{s_{-i} \in \mathcal{S}_{-i}} \left[ \Pi_{k \neq i} \sigma_{k}(s_{k}) \right] \left[ u_{i}(\sigma_i', s_{-i}) - u_{i}(s_i, s_{-i}) \right]. $$
		And this expression is positive for all $\sigma_{-i}$ if and only if $u_i(\sigma_i', s_{-i}) - u_{i}(s_{i}, s_{-i})$ is positive for all $s_{-i}$.
	\end{proof}
\end{proposition} ~\\

\begin{example} ~\
	\begin{figure}[h!] \centering
  				\begin{game}{3}{2}[Player $1$][Player $2$]
   	    			   	 	&	  L    &  R   \\
   	 				U   &    $10, 1$   & $0, 4$  \\
   	 				M   &    $4, 2$   & $4, 3$ \\
   	 				D   &    $0, 5$   & $10, 2$  

   	   				\end{game} $$\Rightarrow \frac{1}{2}U+ \frac{1}{2} D \text{ strictly dominates } M.$$
	\end{figure}
\end{example}

\begin{definition}[Best response]
	The strategy $\sigma_i$ is a \textbf{best response} for player $i$ to her rivals' strategies $\sigma_{-i}$ if:
	$$ u_i(\sigma_i, \sigma_{-i}) \geq u_i(\sigma_i', \sigma_{-i}) $$
	for all $\sigma_i' \in \Delta(\mathcal{S}_i)$. Strategy $\sigma_i$ is never a best response if there is no $\sigma_{-i}$ for which $\sigma_{i}$ is a best response.
\end{definition}

\begin{definition}[Rationalisable Strategies]
	In game $\Gamma_N = [I, \{ \Delta(\mathcal{S}_i) \}, \{ u_i(\cdot) \}]$, the strategies in $\Delta(\mathcal{S}_i)$ that survive the iterated elimination of strategies that are never a best response are known as player $i$'s \textbf{rationalisable strategies}.
\end{definition} ~\newpage

\begin{example} ~\\
		\begin{figure}[h!] \centering
  				\begin{game}{4}{4}[Player $1$][Player $2$]
   	    			   	 	& $b_1$ & $b_2$ & $b_3$ & $b_4$   \\
   	 				$a_1$   &    $0, \underline{7}$   & $2, 5$&    $\underline{7}, 0$   & $0, 1$  \\
   	 				$a_2$   &    $5, 2$   & $\underline{3}, \underline{3}$&    $5, 2$   & $0, 1$ \\
   	 				$a_3$   &    $\underline{7}, 0$   & $2, 5$ &    $0, \underline{7}$   & $0, 1$  \\
					$a_4$   &    $0, \underline{0}$   & $0, -2$ &    $0, \underline{0}$   & $\underline{10}, -1$ 
   	   				\end{game} $$\Rightarrow \frac{1}{2}U+ \frac{1}{2} D \text{ strictly dominates } M.$$
	\end{figure} ~\\
	$\Rightarrow b_4$ is never best response for player 2 and \textit{then} $a_4$ is never best response for player 1. ~\\
	$\Rightarrow \{a_1, a_2, a_3\}$ and $\{ b_1, b_2, b_3 \}$ are the rationalisable strategies in this game.
\end{example}

~\newpage

\section{Nash Equilibrium}
 
~\newpage 
 
\section{Subgame Perfection in Dynamic Games}

~\newpage

\section{Excercises}

\subsection*{Advanced Game Theory - 1. Exercise}

\subsubsection*{Excercise 1.1}
	It holds that $|S| = \Pi_{n=1}^{N} M_{n}$, while:
	$$ S = M_{1} \times \dots \times M. $$

\subsubsection*{Excercise 1.2}	

	\tikzset{
	% Two node styles for game trees: solid and hollow
	solid node/.style={circle,draw,inner sep=1.5,fill=black},
	hollow node/.style={circle,draw,inner sep=1.5}
	}

\begin{figure}[htbp]
	\centering
		\caption*{Extensive form game with imperfect information}
	\begin{tikzpicture}[scale=1.5,font=\footnotesize]
	% Specify spacing for each level of the tree
	\tikzstyle{level 1}=[level distance=15mm,sibling distance=40mm]
	\tikzstyle{level 2}=[level distance=15mm,sibling distance=22mm]
	\tikzstyle{level 3}=[level distance=15mm,sibling distance=12.5mm]
	% The Tree
	\node(0)[hollow node,label=above:{$P1$}]{}
	child{node(1)[solid node]{}
		node[label=below:{$T_{0}$}]{} 
		edge from parent node[left,xshift=-6]{$L$}
	}	
	child{node(2)[solid node]{}
		child{node(4)[solid node,label=right:{}]{} 
			child{node(8)[label=below:{$T_{1}$}]{} edge from parent node[left]{$x$}}
			child{node(9)[label=below:{$T_{2}$}]{} edge from parent node[right]{$y$}}
			edge from parent node[right]{$l$}}
		child{node(5)[solid node,label=right:{}]{} 
			child{node(10)[label=below:{$T_{3}$}]{} edge from parent node[left]{$x$}}
			child{node(11)[label=below:{$T_{4}$}]{} edge from parent node[right]{$y$}}
			edge from parent node[right]{$r$}}
		edge from parent node[left,xshift=-3]{$M$}
	}
	child{node(3)[solid node]{}
		child{node(6)[solid node,label=right:{}]{} 
			child{node(12)[label=below:{$T_{5}$}]{} edge from parent node[left]{$x$}}
			child{node(13)[label=below:{$T_{6}$}]{} edge from parent node[right]{$y$}}
			edge from parent node[right]{$l$}}
		child{node(7)[solid node,label=right:{}]{} 
			child{node(14)[label=below:{$T_{7}$}]{} edge from parent node[left]{$x$}}
			child{node(15)[label=below:{$T_{8}$}]{} edge from parent node[right]{$y$}}
			edge from parent node[right]{$r$}}
		edge from parent node[right,xshift=6]{$R$}
	};
	% information sets
	\draw[dashed,rounded corners=10]($(2) + (-.2,.25)$)rectangle($(3) +(.2,-.25)$);
	\draw[dashed,rounded corners=10]($(4) + (-.2,.25)$)rectangle($(5) +(.2,-.25)$);
	\draw[dashed,rounded corners=10]($(6) + (-.2,.25)$)rectangle($(7) +(.2,-.25)$);
	% specify mover at 2nd information set
	\node at ($(2)!.5!(3)$) {$P2$};
	% specify mover at 3nd information set
	\node at ($(4)!.5!(5)$) {$P1$};
	% specify mover at 4nd information set
	\node at ($(6)!.5!(7)$) {$P1$};
  \end{tikzpicture}
\end{figure}


\begin{enumerate}
	\item Die Strategieräume sind:
	  \begin{align*}
 		S_{1} & = \big\{ (L, x, x), (L, x, y), (L, y, x), (L, y, y), ~\hspace{7cm} \\
 			  & ~\qquad (M, x, x), (M, x, y), (M, y, x), (M, y, y), \\
 			  & ~\qquad (R, x, x), (R, x, y), (R, y, x), (R, y, y) \big\} \\
 			  & = \left\{ S_{1}^{1}, \dotsc, S_{1}^{12} \right\} \\
 		S_{2} & = \left\{ (l), (r) \right\}
 	  \end{align*}
	\item It must hold for $p_i, q_i, r_i \geq 0$ that:
		\begin{align*}
			p_{1} + p_{2} + p_{3} & = 1 \\
			q_{1} + q_{2} & = 1 \\
			r_{1} + r_{2} & = 1
		\end{align*}
		Example of a behaviour strategy: $(p_{1}L + p_{2}M + p_{3}R, q_{1} x + q_{2}y, r_{1}x + r_{2} y)$ \\
		Example of a mixed strategy: $\sum_{i=1}^{12} p_{i} S_{1}^{i}$ \\
		For player 2 there is nothing to show. \\ 
		
		Probability distribution of the outcomes:
		$$ p_{1}, ~ p_{2} \sigma(l) q_{1}, ~ p_{2} \sigma(l) q_{2}, ~ p_{2} \sigma(r) q_{1}, ~ p_{2} \sigma(r) q_{2}, \dotsc $$
		The following mixed strategy of player 1 is realisation equivalent
		$$ \left( p_{1} S^{1} + p_{2} q_{1} S_{1}^{5} + p_{2} q_{2} S_{1}^{7} + p_{3} r_{1} S_{1}^{9} + p_{3} r_{2} S_{1}^{10} \right) $$
		z.z.: ~ $ 1 = p_{1} + p_{2} q_{1} + p_{2} q_{2} + p_{3} r_{1} + p_{3} r_{3}$. Beweis: klar.
	\item Musterlösung im Ilias
\end{enumerate} ~\\

\subsubsection*{Excercise 1.3} ~\\

	\begin{table}[!htbp]
		\centering
	
		\begin{game}{2}{4}[Player 1][Player 2]
	   		   &  LL     &  L & M, & R    \\
	 		U  &  $100, 2$ & $-100, 1$ & $0,0$ & $-100, -100$  \\
	 		D  &  $-100, -100$ & $100, -49$ & $1, 0$ & $100, 2$ \\
		\end{game}
	\end{table}


\subsection*{Advanced Game Theory - 2. Exercise}

\subsubsection*{Excercise 2.1}

$\Sigma_i^N$ set of strategies for player $i$ that remain after $N$ rounds of elimination of never best response strategies. ~\\

Suppose $s_1 \in \Sigma_i^N$ is never best response to any strategy in $\Sigma_{-i}^N$. Suppose we do not delete $s_1$ in round $N+1$. Now $s_1$ will not be a best response to any strategy in $\Sigma_{-i}^{N+1} \subseteq \Sigma_{-i}^{N}$. In particular, it will never be a best response to any strategy in $\Sigma_{-i}^{N+k}$ for $k \geq 1$.
$$ \Rightarrow s_1 \text{ will be deleted in a later round.} $$

\subsubsection*{Excercise 2.2} % todo hier stammt etwas nicht

Suppose $s_i \in S_i$ is stricty dominated by $\sigma_i^{*} \in \Delta(S_i)$. Suppose further that $\sigma_i \in \Delta(S_i)$ plays $s_i^{*}$ with positiv probability $\sigma_i(s_i^*)$ ~\\

\textbf{Claim:} $\sigma_i$ is strictly dominated by $\sigma_i' \in \Delta(S_i)$ which is equivalent to $\sigma_i$ but puts $\sigma(S_i)$ to $\sigma_i^*$.

\begin{proof}
	\begin{align*}
		u_i(\sigma_i, \sigma_{-i}) & = \sum_{s_i \in S_i} \sigma_i(s_i) u_i(s_i, \sigma_{-i}) \\
		& = \sum_{s_i \neq s_i^*} \sigma_i(s_i, u_i(s_i, \sigma_i) + \sigma_i(s_i^*) u_i(s_i^*, \sigma_{-i}) \\
		& < \sum_{s_i \neq s_i^*} \sigma_i(s_i, u_i(s_i, \sigma_i) + \sigma_i(s_i^*) u_i(\sigma_i^*, \sigma_{-i}) \\
		& = u_i(\sigma_i', \sigma_{-i}) \quad \forall \sigma_{-i} \in \Delta(S_{-i})
	\end{align*}
\end{proof}

\subsubsection*{Excercise 2.3}

\begin{enumerate}
	\item keine
	\item keine
	\item $s_1$ (by p.g. $s_2$). Any mixed strategy that contains $s_1$ (by 2.2).
	\item e) f) Musterlösung im Ilias
\end{enumerate}


\subsection*{Advanced Game Theory - 3. Exercise}


\subsubsection*{Excercise 3.1}
  Assumption we make: finite number of pure strategies $\Rightarrow$ there exists a Nash-Equilibrium. ~\\
  
  When a strategy $\sigma_{i}$ is eliminated then so is every strategy that plays $\sigma_{i}$ with positive probability.
  
  $$ S^{\infty}: \text{ set of strategies that survive iterated elimination of strictly dominated strategies.} $$
  
  $\left| S^{\infty} \right| = 1$. ~\\
  
  \textbf{Claim:} If $\left(s_{1}^{*}, \dotsc, s_{I}^{*} \right)$ is a Nash-Equilibrium, then $s^{*} \in S^{\infty}$.
  \begin{proof}
  	Let $\left( s_{1}^{*}, \dotsc, s_{I}^{*} \right)$ be a Nash-Equilibrium and assume $s^{*} \notin S^{\infty}$. Let $i$ be the player whose strategy is eliminated first (in round $k$). ~\\
  	
  	i.e. $\exists \sigma_{i}, \sigma_{i}' \in \Delta\left(S_{i}\right)$:
  	$$ u_{i}(\sigma_{i}, s_{-i}) > u_{i}(\sigma_{i}', s_{-i}) \quad \forall s_{i} \in S_{-i}^{k-1} $$
  	and $\sigma_{i}'$ is played with positiv probability in $s_{i}^{*}$. ~\\
  	
  	Let $s_{i}'$ be derived from $s_{i}^{*}$ with replacing $\sigma_{i}'$ by $\sigma_{i}$.
  	\begin{align*}
  		\Rightarrow \quad u_{i}(s_{i}', s_{-i}^{*}) &= u_{i}(s_{i}^{*},  s_{-i}^{*}) + \underbrace{s_{i}^{*}}_{> 0}(\sigma_{i}')\underbrace{\left[ u_{i}(\sigma_{i}, s_{-i}^{*}) - u_{i}(\sigma_{i}', s_{-i}^{*}) \right]}_{> 0} \\
  		& > u_{i}(s_{i}^{*}, s_{-i}^{*})
  	\end{align*} 
  which contradicts the fact that $s^{*}$ is a Nash-Equilibrium.
  \end{proof}
  
  
\subsubsection*{Excercise 3.2}

Musterlösung

  
\subsubsection*{Excercise 3.3}
\begin{table}[!htbp]
\centering
	
\begin{game}{2}{4}[Player 1][Player 2]
	    &  LL     &  L & M, & R    \\
	 U  &  $100, 2$ & $-100, 1$ & $0,0$ & $-100, -100$  \\
	 D  &  $-100, -100$ & $100, -49$ & $1, 0$ & $100, 2$ \\
\end{game}
\end{table}

\begin{enumerate}
	\item Play $M$, todo: explanation
	\item Pure Nash-Equilibria: $(U, LL)$ and $(D, R)$ ~\\
		Mixed Equilibria: 
		\begin{enumerate}
			\item Player 1 mixes $U$ and $D$ with probabilities $p$ and $1 -p$ respectively.
			\item Player 2 can mix between: $(LL, L), (LL, M), (LL, R), (L, M), (L, R),$
				$$ (M, R), (LL, L, M), (LL, L, R), (LL, M, R), (L, M, R), (LL, L, M, R) $$
		\end{enumerate}
		\textbf{Claim}: Only $(LL, L)$ will lead to a Nash-Equilibrium.
			\begin{proof}[Proof (Using the Proposition after the Definition of Mixed Strategy NE)] ~\\
				Only $(LL, L)$ will lead to a Nash Equilibrium
				\begin{align*}
					u_{2}(LL) = u_{2}(L) ~ & \iff ~ 2p - 100 (1-p) = p - 49 (1-p) \\
					& \iff ~p = \frac{51}{52}
				\end{align*} 
				Therefore: $u_{2}(LL) = u_{2}(L) = \frac{1}{26}$, $u_{2}(M) = 0$, $u_{2}(R) < 0$.
				\begin{align*}
					u_{1}(u) = u_{1}(D)  ~ & \iff ~ 100q - 100(1-q) = -100q + 100(1-q) \\
					& \iff ~ q = \frac{1}{2}
				\end{align*} 
				where $q$ is the probability of Player 2 playing $LL$.~\\
				$$ \Rightarrow ~ \text{ Nash Equilibrium: } \left( \frac{51}{52} U + \frac{25}{26} D, \frac{1}{2} LL + \frac{1}{2} L \right). $$
				Now we have proven that $(LL, L)$ is a Nash Equilibrium. We will subsequently show that no other Nash Equilibrium exists:
				\begin{itemize}
					\item $(LL, M)$: $u_{2}(LL) \overset{!}{=} u_{2}(M) = 0 \iff p = \frac{50}{51}$, but then $u_{2}(L) = \frac{1}{51} > 0$ and hence deviation would result in a higher payout. Therefore $(LL, M)$ is no Nash Equilibrium.
					\item $(LL, R)$: $u_{2}(LL) = u_{2}(R) \iff p = \frac{1}{2}$, but then $u_{2}(LL) = - 49$ and $u_{2}(M) = 0 > - 49$ and again a contradiction to the Nash Equilibrium $(LL, R)$
					\item $(L, M)$, $(L, R)$, $(M, R)$, $(M, L, R)$: choosing on of these strategies we can see in the Normalform representation that Player 1 will always play $D$ $\Rightarrow$ Player 2 plays $R$ without mixing it, hence there is no positiv probability in playing $M$ or $L$.
					\item For the remaining cases four cases the proof follows analogously; we find the necessary probability and show that deviation is enlarging the utility.
				\end{itemize}
			\end{proof}
	\item $M$ is not part of any Nash Equilibrium. However, $M$ is best response to $\frac{1}{2} U + \frac{1}{2}D$ and therefore rationalisable.
	\item Whenever communication is possible, we can even expect $(U, LL)$ or $(D, R)$ as outcome as both players would profit.
\end{enumerate}

\chapter{Kooperative Spiele}

\section{Der Kern}

~\newpage

\section{Der Shapley-Wert}

~\newpage

\section{Einfache Spiele}

~\newpage

\section{Konvexe Spiele}

~\newpage

\section{Übungen}

\subsection*{Advanced Game Theory - 4. Exercise}

\subsubsection*{Aufgabe 4.1}

Gegeben sei ein Drei-Personen-Abstimmungsspiel $\Gamma_{C} = [N, v]$ mit $N = \{1, 2, 3\}$, in dem jeder Spieler genau eine Stimme hat und in dem anhand der Einfachen-Mehrheit-Regel über die Aufteilung $x$ eines Kuchens auf die drei Personen entschieden werden soll, wobei $x = (x_1, x_2, x_3) \in \mathbb{R}^{3}$, $x_i \neq 0$ für alle $i \in N$ und $\sum_{i} x_{i} \leq i$. Der individuelle Nutzen eines jeden Spielers ist gleich dem Anteil am Kuchen, den er erhält, d.h. $u_i(x_i) = x_i$ für $i \in N$.

\begin{enumerate}
	\item Bestimmen Sie die charakteristischen Funktionswerte v(K) aller Koaliationen $K \subseteq N$.
		\begin{proof}
			$$ v(\{ 1 \}) = 0, \quad v(\{ 2 \}) = 0, \quad v(\{ 3 \}) = 0, \quad v(\{ 1, 2,3 \}) = 1  $$
			$$ v(\{ 1, 2 \}) = 1, \quad  v(\{ 2, 3 \}) = 1, \quad v(\{ 1, 3 \}) = 1 $$
		\end{proof}
	\item Bestimmen Sie den Kern $C(\Gamma_C)$ und den Shapley-Wer $\Phi(\Gamma_C)$.
		\begin{proof}
			Den Kern $C(\Gamma_C)$ erhält man, indem man einer Aufteilung $x_1, x_2, x_3$ das folgende Gleichungssystem als Randbedingungen mitgibt:
			\begin{align*}
				x_{1} + x_{2} + x_{3} = 1 & = v(\{1, 2, 3 \}) \\
				x_{1} + x_{3} \geq 1 & = v(\{1, 3 \}) \\
				x_{2} + x_{3} \geq 1 & = v(\{ 2, 3 \}) \\
				x_{1} + x_{2} \geq 1 & = v(\{ 1, 2 \}) \\
				x_{3} \geq 0 & = v(\{ 3 \}) \\
			    x_{2} \geq 0 & = v(\{ 2 \}) \\
				x_{1} \geq 0 & = v(\{ 1 \})
			\end{align*}
			Setzen wir die Gleichungen 2 - 4 ineinander ein, so erhalten wir:
			$$ x_{1} \geq 1 - x_{2}, \quad x_{3} \geq 1 - x_{2} $$
			$$ 1 - x_{2} + 1 - x_{2} \geq 1 \iff x_{2} \geq \frac{1}{2} $$
			Aus Symmetrie (oder einfach Wiederholung der obigen Schritte für $x_{1}$ und $x_{2}$) erhalten wir:
			$$ x_{1}, x_{2}, x_{3} \geq \frac{1}{2}. $$
			Allerdings bedeutet dies:
			\begin{equation*}
				\frac{1}{2} + \frac{1}{2} + \frac{1}{2} \leq x_{1} + x_{2} + x_{3} = 1, \qquad \lightning
			\end{equation*} 
			d.h. $C(\Gamma_C) = \emptyset$. 
			Für den Shapley-Wert betrachten wir folgendes:
			\begin{center}
    			\begin{tabular}{| c | c | c | c |}
   					\hline
    					Reihenfolge/Marg. Beitrag &  Sp. 1 & Sp. 2 & Sp. 3  \\ 
    						\hline
    					$1, 2, 3$ & $0$ & $1$ & $0$  \\ 
    						\hline
    					$1, 3, 2$ & $0$ & $0$ & $1$  \\
    						\hline
    					$2, 1, 3$ & $1$ & $0$ & $0$  \\
       						\hline
    					$2, 3, 1$ & $0$ & $0$ & $1$  \\
      						\hline
    					$3, 1, 2$ & $1$ & $0$ & $0$  \\
      						\hline
    					$3, 2, 1$ & $0$ & $1$ & $0$  \\
      						\hline \hline
    					$\phi_{i}(\Sigma_{C}) = \Sigma$  & $2$ & $2$ & $2$  \\
    				\hline
   				 \end{tabular}
    		\end{center}
    		d.h. $\Phi(\Sigma_{C}) = \left(\frac{2}{6}, \frac{2}{6}, \frac{2}{6} \right)$.
		\end{proof}
	\item Lösen Sie die Teilaufgaben a) und b) unter der Bedingung, dass Koalitionen, die sowohl Spieler 2 als auch Spieler 3 enthalten, nicht gebildet werden.
		\begin{proof}
			Die Randbedingung ändern sich wie folgender Maßen:
			\begin{align*}
				x_{1} + x_{2} + x_{3} = 1 & = v(\{1, 2, 3 \}) \\
				x_{1} + x_{3} \geq 1 & = v(\{1, 3 \}) \\
				x_{2} + x_{3} \geq 0 & = v(\{ 2, 3 \}) \\
				x_{1} + x_{2} \geq 1 & = v(\{ 1, 2 \}) \\
				x_{3} \geq 0 & = v(\{ 3 \}) \\
			    x_{2} \geq 0 & = v(\{ 2 \}) \\
				x_{1} \geq 0 & = v(\{ 1 \})
			\end{align*}
			d.h. die eine/zwei Randbedingungen werden trivial. Der Kern besteht also aus
			$$ x_{3} \geq 1 - x_{1}, \quad x_{2} \geq 1 - x_{1} $$
			$$ \Rightarrow x_{1} = 1 $$
			D.h. $C(\Gamma_{C}) = \{ (1, 0, 0) \}$. Der Shapely-Wert lässt sich wieder über folgendes bestimmen
			\begin{center}
    			\begin{tabular}{| c | c | c | c |}
   					\hline
    					Reihenfolge/Marg. Beitrag &  Sp. 1 & Sp. 2 & Sp. 3  \\ 
    						\hline
    					1, 2, \deleted{\color{red}{3}} & $0$ & $1$ & $0$  \\ 
    						\hline
    					1, 3, \deleted{\color{red}{2}} & $0$ & $0$ & $1$  \\
    						\hline
    					2, 1, \deleted{\color{red}{3}} & $1$ & $0$ & $0$  \\
       						\hline
    					2, \deleted{\color{red}{3, 1}} & $0$ & $0$ & $0$  \\
      						\hline
    					3, 1, \deleted{\color{red}{2}} & $1$ & $0$ & $0$  \\
      						\hline
    					3, \deleted{\color{red}{2, 1}} & $0$ & $0$ & $0$  \\
      						\hline \hline
    					$\phi_{i}(\Sigma_{C}) = \Sigma$  & $2$ & $1$ & $1$  \\
    				\hline
   				 \end{tabular}
    		\end{center}
    		d.h. $\Phi(\Sigma_{C}) = \left(\frac{2}{c}, \frac{1}{c}, \frac{1}{c} \right)$; die Frage bleibt aber, welchen Wert $c$ annehmen muss. Mein Tipp wäre $\frac{v(N)}{\sum \phi_{i}(\Sigma_{C})}$\footnote{Scheint mir nicht ganz konsistent mit der Vorlesung $(1/n!)$ zu sein}. Laut Musterlösung gilt $c = 4$ was konsistent mit meiner Vermutung wäre.
		\end{proof}
	\item Lösen Sie die Teilaufgaben a) und b) für das Drei-Personen-Abstimmungsspiel $\Gamma_C = [N, v]$ mit $N = \{1, 2, 3\}$ in dem Spieler 1 ein Stimmengewicht von 60\% und Spieler 2 und 3 von jeweils 20\% besitzen und Entscheidungen anhand der Zweidrittel-Mehrheit-Regel (qualifizierte Mehrheit) getroffen werden.
	  \begin{proof}
	  ~\\
		\begin{enumerate} 
			\item Bestimmen Sie die charakteristischen Funktionswerte v(K) aller Koaliationen $K \subseteq N$.
			$$ v(\{ 1 \}) = 0, \quad v(\{ 2 \}) = 0, \quad v(\{ 3 \}) = 0, \quad v(\{ 1, 2, 3 \}) = 1  $$
			$$ v(\{ 1, 2 \}) = 1, \quad  v(\{ 1, 3 \}) = 1, \quad v(\{ 2, 3 \}) = 0 $$
			\item Bestimmen Sie den Kern $C(\Gamma_C)$ und den Shapley-Wer $\Phi(\Gamma_C)$. \\
			
			Den Kern $C(\Gamma_C)$ erhält man, indem man einer Aufteilung $x_1, x_2, x_3$ das folgende Gleichungssystem als Randbedingungen mitgibt:
			\begin{align*}
				x_{1} + x_{2} + x_{3} = 1 & = v(\{1, 2, 3 \}) \\
				x_{1} + x_{3} \geq 1 & = v(\{1, 3 \}) \\
				x_{1} + x_{2} \geq 1 & = v(\{ 1, 2 \}) \\
				x_{2} + x_{3} \geq 0 & = v(\{ 2, 3 \}) \\
				x_{3} \geq 0 & = v(\{ 3 \}) \\
			    x_{2} \geq 0 & = v(\{ 2 \}) \\
				x_{1} \geq 0 & = v(\{ 1 \})
			\end{align*}
			d.h. $x_{3} \geq 1 - x_{1}$, $x_{2} \geq 1- x_{1}$.
			$$ 1 \geq 1 - x_{3} + x_{2} + x_{3} \iff x_{2} = 0$$
			$$ 1 \geq 1 + x_{3} + x_{2} - x_{2} \iff x_{3} = 0$$
    		d.h. $\Phi(\Sigma_{C}) = \left(1, 0, 0 \right)$. Schneller geht das auch, durch das Theorem, dass bei einem Veto-Spieler alle anderen die Auszahlung $0$ erhalten müssen. Für den Shapley-Wert betrachten wir:
  			\begin{center}
    			\begin{tabular}{| c | c | c | c |}
   					\hline
    					Reihenfolge/Marg. Beitrag &  Sp. 1 & Sp. 2 & Sp. 3  \\ 
    						\hline
    					$1, 2, 3$ & $0$ & $1$ & $0$  \\ 
    						\hline
    					$1, 3, 2$ & $0$ & $0$ & $1$  \\
    						\hline
    					$2, 1, 3$ & $1$ & $0$ & $0$  \\
       						\hline
    					$2, 3, 1$ & $1$ & $0$ & $0$  \\
      						\hline
    					$3, 1, 2$ & $1$ & $0$ & $0$  \\
      						\hline
    					$3, 2, 1$ & $1$ & $0$ & $0$  \\
      						\hline \hline
    					$\phi_{i}(\Sigma_{C}) = \Sigma$  & $4$ & $1$ & $1$  \\
    				\hline
   				 \end{tabular}
    		\end{center}
    		d.h. $\Phi(\Sigma_{C}) = \left(\frac{4}{c}, \frac{1}{c}, \frac{1}{c} \right)$, $c = 4$(?).
		\end{enumerate}
	\end{proof}
\end{enumerate}

\subsubsection*{Aufgabe 4.2}
 
Gegeben sei folgende Auszahlungstabelle eines Zwei-Personen-Spiels in Normalform 
	
\begin{figure*}[h!]
  \begin{center}
	\begin{game}{2}{2}[~][~]
	    	  &   $s_{21}$   &   $s_{22}$   \\
	 $s_{11}$ &  $3,      3$ & $0, \alpha$  \\
	 $s_{12}$ &  $\alpha, 0$ & $1, 1$     
	\end{game}
  \end{center}
\end{figure*}

Beschreiben Sie jeweils für $\alpha = 5$ und $\alpha = 7$ das korrespondierende Koalitionsspiel $\Gamma_C = [N, v]$ und bestimmen Sie den Kern $C(\Gamma_C)$.

	\begin{proof}
		Wir haben das Spiel gegeben durch $N = \{1, 2\}$ und $v \colon P(N) \rightarrow \R$.
		
		Angenommen $v$ ist superadditiv, und da wir wissen, dass dieses Spile symmetrisch ist, gilt:
		\begin{enumerate}
			\item $\alpha = 5$ \\
				$$ v(N) = \max_{i,j,k,j \in N} u(s_{ij},s_{kj}) = \max_{i,j,k,j \in N} \left( u_{1}(s_{ij},s_{kj}) + u_{2}(s_{ij},s_{kj}) \right) = 3 + 3 = 6, $$ 
				$$ v(\{1\}) = v(\{2\}) = \min_{i,j,k,j \in N} u_{1/2}(s_{ij},s_{kj}) = 1  $$
		
				Um den Kern zu bestimmen, betrachte:
				\begin{align*}
					x_{1} + x_{2} = 6 & = v(\{1, 2, 3 \}) \\
			    	x_{2} \geq 1 & = v(\{ 2 \}) \\
					x_{1} \geq 1 & = v(\{ 1 \})
				\end{align*}
				$$ \Rightarrow C(\Gamma_{C}) = \{ x_{1}, x_{2} \colon x_{1}, x_{2} \geq 1, x_{1} + x_{2} = 6 \} \neq \emptyset $$ 
			\item $\alpha = 6$ \\
				$$ v(N) = \max_{i,j,k,j \in N} u(s_{ij},s_{kj}) = \max_{i,j,k,j \in N} \left( u_{1}(s_{ij},s_{kj}) + u_{2}(s_{ij},s_{kj}) \right) = 7 + 0 = 0 + 7 = 7, $$ 
				$$ v(\{1\}) = v(\{2\}) = \min_{i,j,k,j \in N} u_{1/2}(s_{ij},s_{kj}) = 1  $$
		
				Um den Kern zu bestimmen, betrachte:
				\begin{align*}
					x_{1} + x_{2} = 7 & = v(\{1, 2, 3 \}) \\
			    	x_{2} \geq 1 & = v(\{ 2 \}) \\
					x_{1} \geq 1 & = v(\{ 1 \})
				\end{align*}
				$$ \Rightarrow C(\Gamma_{C}) = \{ x_{1}, x_{2} \colon x_{1}, x_{2} \geq 1, x_{1} + x_{2} = 7 \} \neq \emptyset $$
		\end{enumerate}
	\end{proof}

\subsubsection*{Aufgabe 4.3}

Ein Kleintierzüchterverein hat sieben Mitglieder: zwei Meerschweinchenzüchter $M_1$ und $M_2$, zwei Taubenzüchter $T_1$ und $T_2$ und drei Hasenzüchter $H_1$, $H_2$ und $H_3$. Entscheidungen werden mit einfacher Mehrheit gefällt.

  \begin{enumerate}
 	\item Beschreiben Sie unter der Bedingung, dass die Mitglieder einer Zuchtgruppe stets einheitlich abstimmen, das Koalitionsspiel $\Gamma_C = [N, v]$ für die drei unabhängingen Spieler in Form der drei Zuchtgruppen $M = \{M_1,M_2\}$, $T = \{T_1, T_2\}$ und $H = \{H_1,H_2,H_3\}$, also $N = \{M, T, H\}$, und berechnen Sie die Shapley-Werte für $M$, $T$ und $H$.
 		\begin{proof}
 			Es gilt $\Gamma_C = [N, v]$, wobei $N = \{ M, T, H \}$ und $v \colon P(N) \rightarrow \N$ mit:
 			$$ v(N) = 1, ~ v(\{M, T\}) = 1, ~ v(\{T, H\}) = 1, ~ v(\{M,H\}) = 1,$$
 			$$  v(\{M\}) = v(\{T\}) = v(\{H\}) = 0. $$
			Für den Shapley-Wert betrachten wir:
  			\begin{center}
    			\begin{tabular}{| c | c | c | c |}
   					\hline
    					Reihenfolge/Marg. Beitrag & M & T & H \\ 
    						\hline
    					$M, T, H$ & $0$ & $1$ & $0$  \\ 
    						\hline
    					$T, H, M$ & $0$ & $0$ & $1$  \\
    						\hline
    					$T, M, H$ & $1$ & $0$ & $0$  \\
       						\hline
    					$H, T, M$ & $0$ & $1$ & $0$  \\
      						\hline
    					$H, M, T$ & $1$ & $0$ & $0$  \\
      						\hline
    					$M, H, T$ & $0$ & $0$ & $1$  \\
      						\hline \hline
    					$\phi_{i}(\Sigma_{C}) = \Sigma$  & $2$ & $2$ & $2$  \\
    				\hline
   				 \end{tabular}
    		\end{center}
    		d.h. $\Phi(\Sigma_{C}) = \left(\frac{2}{c}, \frac{2}{c}, \frac{2}{c} \right)$, $c = 6$(?).
 		\end{proof}
	\item Eines Tages zerstreiten sich die drei Hasenzüchter, was dazu führt, dass sie die Hasenkoalition auflösen und in Abstimmungen einzeln auftreten. Die Meerschweinchenzüchter und Taubenzüchter stimmen weiterhin einheitlich ab. Wie lauten die Ergebnisse von Teilaufgabe a) für die fünf unabhängingen Spieler $M$, $T$, $H_1$, $H_2$ und $H_3$. Vergleichen Sie die Shapley-Werte mit denen von Teilaufgabe a). Was fällt auf?
 		\begin{proof}
 			Es gilt $\Gamma_C = [N, v]$, wobei $N = \{ M, T, H_{1}, H_{2}, H_{3} \}$ und $v \colon P(N) \rightarrow \N$ mit:
 			$$ v(\{T, H_{1}, H_{2}, H_{3} \}) = 1, ~ v(\{M, H_{1}, H_{2}, H_{3} \}) = 1,  ~ v(\{T, H_{i}, H_{j} \}) = 1, ~ v(\{M, H_{i}, H_{j} \}) = 1,$$
 			$$  v(\{M\}) = v(\{T\}) = v(\{ H_{i} \}) = v(\{ H_{i}, H_{j} \}) = v(\{ H_{1}, H_{2}, H_{3} \}) = 0. $$
 			$$ v(N) = 1, ~ v(\{M, T\}) = 1, ~v(\{T, H_{i}\}) = 0, ~ v(\{M, H_{i}\}) = 0$$
			Aus Symmetrie-Gründen können wir den Shapley-Wert für z.B. $T$ berechnen über:
  			\begin{center}
    			\begin{tabular}{| c | c | c | c |}
   					\hline
    					Reihenfolge & Marg. Beitrag von T \\ 
    						\hline
    					$M, T, \pi(H_{1}, H_{2}, H_{3})$ & $6$   \\ 
    						\hline
    					$M, H_{i}, T, \pi(H_{j}, H_{k})$ & $3 \cdot 2 = 6$  \\
    						\hline
    					$H_{i}, M, T, \pi(H_{j}, H_{k})$ & $3 \cdot 2 = 6$   \\
       						\hline
    					$\pi(H_{i}, H_{j}), T, M, H_{k}$ & $3 \cdot 2 = 6$   \\
      						\hline
    					$\pi(H_{i}, H_{j}), T, H_{k}, M$ & $3 \cdot 2 = 6$ \\
      						\hline
    					$\pi(H_{1}, H_{2}, H_{3}), M, T$  & $6$ \\
      						\hline \hline
    					$\phi_{T}(\Sigma_{C}) = \Sigma$  & $36$  \\
    				\hline
   				 \end{tabular}
    		\end{center}
    		d.h. $\Phi_{T}(\Sigma_{C}) = \frac{36}{n!} = \frac{36}{120} = \frac{3}{10}$. Eben aus Symmetrie-Gründen gilt: $\Phi_{T}(\Sigma_{C}) = \Phi_{M}(\Sigma_{C})$. \\ 
    		Schließlich gilt wieder aus Symmetriegründen:
    		$$ \Phi_{H_{i}}(\Sigma_{C}) = \frac{1 - \Phi_{T}(\Sigma_{C}) - \Phi_{M}(\Sigma_{C})}{3} = \frac{1 - 0,\overline{3} - 0,\overline{3}}{3} = 0,1\overline{3}, \quad \forall i \in \{1, 2, 3\}. $$
    		Es fällt auf, dass in der Summe die Shapley Werte der Hasen höher ist, als in der a). Dies ist der Kritikpunkt am Shapley-Wert.
 		\end{proof}
  \end{enumerate}

\chapter{Evolutionäre Spieltheorie}

  
\section{Spiele in Normalform}
Für symmetrische Spiele:
$$ A = \left( a_{ij} \right) \quad i = 1, \dotsc, m_{i}, ~ j = 1, \dotsc, m_{j} $$
d.h.
% game table 
\begin{description}
	\item $N$: Spielermenge $|N| = n$
	\item $\Sigma_{i}$: Menge der reinen Strategien von $i \in N$, $\left| \Sigma_{i} \right| = m_{i}$, $\sigma_{i} \in \mathcal{E}_{i}$.
	\item $S_{i}$: Menge der gemischten Strategien von $i \in N$
		\begin{description}
			\item $S_{i} = \left\{ \right\}$.
			\item $s_{ij} = \mathds{P}(\sigma_{ij})$.
		\end{description}
\end{description}
  
\begin{definition}[Trägermenge] Wir definieren die Trägermenge für jeden Spieler $i \in N$:
	$$ C(S_{i}) = \left\{ \sigma_{ij} \in \Sigma_{i} : s_{ij} > 0 \right\}, $$
	als die Menge der Strategien die mit positiver Wahrscheinlichkeit gespielt werden.
\end{definition}  

\begin{definition}[Beste-Antwort-Menge] Sei
	$$ B_{i}(s_{-i}) = \left\{ \sigma_{j} \in \Sigma_{i} : H(\sigma_{ij}, s_{-i}) = \max_{\sigma_{ik \in \Sigma_{i}}} H(\sigma_{ik}, s_{-i}) \right\} $$ 
	$H$ bezeichne pay-off-Funktion ~\\
	$$ \hat{H}(s_{-i}) \coloneqq \max_{\sigma_{il} \in \Sigma_{i}} H(\sigma_{ik}, s_{-i}) $$
\end{definition}
  
  
\begin{beispiel*}
	$\sigma_{ij} \in B_{i}(S_{-i})$ und $\sigma_{ik} \in B(S_{-i}) \Rightarrow$ alle $s_{i} \in S_{i}$ mit
	$$ C(S_{i}) = \{ \sigma_{ij}, \sigma_{ik} \} $$	
	sind auch beste Antwort, denn
	$$ s_{ij} H(\sigma_{j}, s_{-i}) + s_{ik} H(\sigma_{ik}, s_{-i}) = (s_{ij} + s_{ik}) \hat{H}(s_{-i}) = \hat{H}(s_{-i}). $$
\end{beispiel*}

Sei $s^{*} = \left( s_{1}^{*}, \dotsc, s_{n}^{*} \right)$ ein Nash-Gleichgewicht. Mit $s_{i}^{*} = (s_{i1}^{*}, \dotsc s_{im_{i}}^{*})$ gilt
$$ C(s_{i}^{*}) \subseteq B_{i}(s_{-i}^{*}) $$
 \textit{Hinreichend? Ja! Proposition Slide 36 (AGT Teil 1)}.


\begin{unnamedtheorem}[Grundannahmen der evolutionären Spieltheorie]
	\begin{enumerate}
		\item große Population
		\item Population ist monomorph
		\item random matching
		\item Wettstreit (Spiel) ist statisch und symmetrisch
			$$ \rightarrow \text{ symmetrisches Spiel in Normalform mit zwei Spielern}. $$
		\item Auszahlung entspricht der \enquote{biologischen Fitness} ($\phi$ Anzahl Nachkommen)
		\item Reproduktion ist asexuell und die von den Eltern gewählt Strategie wird unverändert an die Nachkommen vererbt (nur Selektion, keine Mutation).
	\end{enumerate}
\end{unnamedtheorem} 
 

\begin{unnamedtheorem}[Symmetrisches 2-Personenspiel in Normalform]
	Spieler müssen nicht unterschieden werden $\Rightarrow$ Strategieraum:
	$$ S = \{ s\in \R^{m} : \sum_{i=1}^{m} s_{i} = 1, s_{i} \geq 0, i = 1, \dotsc, m \} $$
\end{unnamedtheorem}
  
 
\begin{definition}[Evolutionär stabile Strategie, ESS]
	Eine Strategie $p \in S$ hei{\ss}t evolutionär stabil, wenn
	\begin{enumerate}
		\item $H(p,p) \geq H(q, p)$ für alle $q \in S$ (Gleichgewichtsbedingung)
		\item Für alle $q \in S \setminus \{ p \}$ mit $H(q, p) = H(p, p)$ gilt: $H(p, q) > H(q, q)$ (Stabilitätsbedingung)
	\end{enumerate}
\end{definition}  


\begin{unnamedtheorem}[Eigenschaften von evolutionär stabilen Strategien] ~\
	\begin{itemize}
		\item Ist $p \in S$ eine evolutionär stabile Strategie, dann bildet $(p, p)$ ein symmetrisches Nash-Gleichgewicht
		\item Jede $2 \times 2$-Matrix $A = \begin{pmatrix}
			a_{11} & a_{12} \\ a_{21} & a_{22}
		\end{pmatrix}$ mit $H(p,p) = p'Ap$ sodass $a_{11} \neq a_{21}$ und / oder $a_{12} = a_{22}$, besitzt eine ESS
		Ist $(p,p)$ ein striktes NGG, dann ist $p$ eine ESS. Im strikten NGG $(p, p)$ gilt $C(p) = B(p)$. Ein striktes NGG ist immer ein Gleichgewicht in reinen Strategien. Beispiel:
   \begin{game}{2}{2}[~][]
   	    &  ~      &  ~     \\
   	 ~  &    $3, 3$      & $2, 0$  \\
   	  	&  $0, 2$ & $4, 4$\\
   \end{game}
	\item Im Normalformspielen mit $m \times m$-Matrizen $a$ mit $m \geq 3$ existieren entweder endlich viele ESS keine.
	\end{itemize}
\end{unnamedtheorem}
  
\begin{unnamedtheorem}[Allgemein gilt]
	Ist $p$ ESS $\Rightarrow \neg \exists \sigma \in C(p)$ mit $\sigma \in C(S^{*})$ für $s^{*} \neq q$ ist Nash-Gleichgewicht
\end{unnamedtheorem} 

$\Rightarrow \#$ ESS $\leq \left| \Sigma \right|$ - Gleichheit nur, fall es kein ESS in gemischten Strategien gibt.  
  
\section{Aufgaben}

\subsubsection*{Aufgabe 5.2}  
  \begin{enumerate}[label=\alph*\upshape)] \setcounter{enumi}{1}
  	\item Gegeben sei:
      $$ \begin{game}{2}{2}[~][~]
   	    &  $\sigma_{1}$      &  $\sigma_{2}$     \\
   	 $\sigma_{1}$  &    0, 0      & 0, 0  \\
   	 $\sigma_{2}$ 	&  0, 0 & 1, 1 \\
   \end{game}$$ $\Rightarrow A = \begin{pmatrix} 0 & 0 \\ 0 & 1 \end{pmatrix} \text{ (Gegenbeispiel)}$
   $$ \sigma^{*} = (\sigma_1, \sigma_1) , \sigma^{**} = (\sigma_2, \sigma_2) $$
  \end{enumerate}
    
  \begin{enumerate}[label=\alph*\upshape)]
  	\item Angenommen $p \in S$ ist ESS  und wird von $q \in S$ schwach dominiert
  		$$ \Rightarrow H(q, z) \geq H(p, z) \quad \forall z \in S $$
  		\begin{align*}
  			\Rightarrow & H(p,p) = H(q, p) \quad \text{ Bedingung 1: ok} \\
  			\Rightarrow & H(p,q) \leq H(q, q) \quad \text{ Bedingung 2: verletzt} \\
  		\end{align*} \setcounter{enumi}{2}
  	\item klar!
  \end{enumerate}
   
\subsubsection*{Aufgabe 5.1 (Hawk-Dove-Game / Falke-Taube-Spiel)}   
	$$
		\begin{game}{2}{2}[~][~]
   	   			&  $F$      &  $T$     \\
   	 $F$  &    $\frac{v-c}{2}$, $\frac{v-c}{2}$      & $v, 0$  \\
   	 $T$ 	&  $v, 0$ & $\frac{v}{2}$, $\frac{v}{2}$ \\
   \end{game}$$
   $A = \begin{pmatrix}
   	\frac{v-c}{2} & v \\ 0 & \frac{v}{2}
   \end{pmatrix}$, $c > v > 0$
   \begin{itemize}
   	\item es existiert keine dominante Strategie
   	\item es existiert kein symmetrisch Nash-Gleichgewicht in reinen Strategien
   	\item $(F, T)$, $(T, F)$ sind strikte Nash-Gleichgewichte
   \end{itemize}
   Interpretation: Recource $v$, Tauben teilen friedlich, Falken vertielgt Zaube, Falken kämpfen $\rightarrow$ neg, outcome für beide. ~\\
   $p = (p_F, p_T)$
  $$	H(F, p) \overset{!}{=} H(T, p) \overset{!}{=} H(p,p) $$
   $$	\begin{rcases} H(F, p) & = p_{F} \frac{v-c}{2} + p_T v \\ H(T, p) & = p_F 0 + p_T \frac{v}{2} \end{rcases} \xrightarrow[]{p_F + p_T =1} p_F = \frac{v}{c}, p_T = 1 - \frac{v}{c} $$
   ist das einzige symmetrisch Nas-Gleichgewicht, kein triviales Spiel $\Rightarrow \exists ESS$
   $$ \Rightarrow \left( \frac{v}{2}, 1 - \frac{v}{2} \right) \text{ ist ESS} $$
   oder man rechnet nach $H(F,p) = H(t, p) = H(p,p)$ ~\\
   $\Rightarrow z.z. H(p, F) > H(F, F)$, $H(p, T) > H(T, T)$

\subsubsection*{Aufgabe 5.3}
$$A = \begin{pmatrix} 1 & 1 & 0 \\ 1& 1 & 1 \\ 0 & 1 & 1 \end{pmatrix}$$   
\begin{enumerate}
	\item Nash-Gleichgewicht in reinen Strategien:
		$$ (x,x), ~(x,y), ~(y,x), ~(y,z), ~(z, y),  ~(z,z) $$
		Trivial: $C(A) = \{ A \}$, $A \in \{ x,y,z\}$
		$$ B(x) = \{x,y\}, ~B(y) = \{x, y,z\}, ~B(z) = \{ y, z \} $$
		$$ \Rightarrow C(\cdot) \underset{\not-}{\subset} B(\cdot) $$
		Nash-Gleichgewicht in gemischten Strategien (nur sym.)
		$$ S^{*} = \{ (s_x, s_y, 0) : s_x \in (0, 1), s_y = 1 - s_x \} \quad C(S^{*}) = \{x, y\} $$
		$$ S^{**} = \{ (0, s_y, s_z) : s_y \in (0, 1), s_z = 1 - s_y \} \quad C(S^{**}) = \{y, z\} $$
		$B(S^{*}) = \{ x, y \}$, $B(S^{**}) = \{y, z \}$
	\item Angenommen $p \in S$ mit $p_x \in [0, 1]$ und $p_y = 1 - p_x$ ist ein ESS
		\begin{description}
			\item Bedingung 1: $\checkmark$
			\item Bedingung 2: $H(x, p) = H(p, p) = 1$ mit $p_x < 1$
				$$ \Rightarrow H(p, x) > H(x, x) \text{ Widerspruch!} $$
				analog in anderen Fällen $\Rightarrow $ ESS existiert nicht.
		\end{description}
\end{enumerate}


\begin{erinnerung}
	Ist $p$ ESS $\Rightarrow \neg \exists \sigma \in C(p)$ mit $\sigma \in C(S^{*})$ für $s^{*} \neq q$ ist Nash-Gleichgewicht.

	$\Rightarrow \#$ ESS $\leq \left| \Sigma \right|$ - wobei Gleichheit nur gilt, fall es kein ESS in gemischten Strategien gibt.
\end{erinnerung}
   


\newpage


\phantomsection \appendix \pagenumbering{Roman}  \cftaddtitleline{toc}{chapter}{Appendix}{}
\cftaddtitleline{toc}{section}{Übungen}{I}




\printindex

\end{document}