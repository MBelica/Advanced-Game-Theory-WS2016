\documentclass[12pt]{extreport} % Schriftgröße: 8pt, 9pt, 10pt, 11pt, 12pt, 14pt, 17pt oder 20pt

%% Packages
\usepackage{scrextend}
\usepackage{amssymb}
\usepackage{amsthm}
\usepackage{booktabs}
\usepackage{chngcntr}
\usepackage{cmap}
\usepackage{color}
\usepackage{enumitem}
\usepackage{hyperref}
\usepackage{ulem}
\usepackage{lmodern}
\usepackage{makeidx}
\usepackage{dsfont}
\usepackage{mathtools}
\usepackage{xpatch}
\usepackage{pgfplots}
\pgfplotsset{compat=1.7}
\usetikzlibrary{calc}	
\usetikzlibrary{matrix}	
\usepackage[margin = 1in]{geometry} % Margins  
\usepackage{setspace} % Setting the spacing between lines
\usepackage{amsthm, amsmath, amsfonts, mathtools, amssymb} % Math packages 
\usepackage{sgame, tikz} % Game theory packages
\usetikzlibrary{trees, calc} % For extensive form games
\usepackage{subfig} % Manipulation and reference of small or sub figures and tables

% Language Setup (Deutsch)
\usepackage[utf8]{inputenc} 
\usepackage[T1]{fontenc} 
\usepackage[ngerman]{babel}

\usepackage{csquotes}
% Options
\makeatletter%%  
  % Linkfarbe, {0,0.35,0.35} für Türkis, {0,0,0} für Schwarz, {1,0,0} für Rot, {0,0,0.85} für Blau
  \definecolor{linkcolor}{rgb}{0,0.35,0.35}
  % Zeilenabstand für bessere Leserlichkeit
  \def\mystretch{1.2} 
  % Publisher definieren
  \newcommand\publishers[1]{\newcommand\@publishers{#1}} 
  % Enumerate im 1. Level: \alph für a), b), ...
  \renewcommand{\labelenumi}{\alph{enumi})} 
  % Enumerate im 2. Level: \roman für (i), (ii), ...
  \renewcommand{\labelenumii}{(\roman{enumii})}
  % Zeileneinrückung am Anfang des Absatzes
  \setlength{\parindent}{0pt} 
  % Verweise auf Enumerate, z.B.: 3.2 a)
  \setlist[enumerate,1]{ref={\thesatz ~ \alph*)}}
  % Für das Proof-Environment: 'Beweis:' anstatt 'Beweis.'
  \xpatchcmd{\proof}{\@addpunct{.}}{\@addpunct{:}}{}{} 
  % Nummerierung der Bilder, z.B.: Abbildung 4.1
  \@ifundefined{thechapter}{}{\def\thefigure{\thechapter.\arabic{figure}}} 
  % Chapter-Nummerierung beginnen bei (-1):
  \setcounter{chapter}{14}
\makeatother%

% Meta Setup (Für Titelblatt und Metadaten im PDF)
\title{Höhere Mathematik II}
\author{G. Herzog, Ch. Schmoeger}
\date{Sommersemester 2017}
\publishers{Karlsruher Institut für Technologie}

%% Math. Definitionen
\newcommand{\C}{\mathbb{C}}
\newcommand{\N}{\mathbb{N}}
\newcommand{\Q}{\mathbb{Q}}
\newcommand{\R}{\mathbb{R}}
\newcommand{\Z}{\mathbb{Z}}

%% Theorems (unnamedtheorem = Theorem ohne Namen)
\newtheoremstyle{named}{}{}{\normalfont}{}{\bfseries}{:}{0.25em}{#2 \thmnote{#3}}
\newtheoremstyle{nnamed}{}{}{\normalfont}{}{\bfseries}{:}{0.25em}{\thmnote{#3}}
\newtheoremstyle{itshape}{}{}{\itshape}{}{\bfseries}{:}{ }{}
\newtheoremstyle{normal}{}{}{\normalfont}{}{\bfseries}{:}{ }{}
\renewcommand*{\qed}{\hfill\ensuremath{\square}}

\theoremstyle{named}
\newtheorem{unnamedtheorem}{Theorem} \counterwithin{unnamedtheorem}{chapter}
\theoremstyle{nnamed}
\newtheorem*{unnamedtheorem*}{Theorem} 

\theoremstyle{itshape}
\newtheorem{satz}[unnamedtheorem]{Satz} 
\newtheorem*{definition}{Definition}
\newtheorem{hilfssatz}[unnamedtheorem]{Hilfssatz}
\newtheorem*{hilfssatz*}{Hilfssatz}

\theoremstyle{normal}
\newtheorem{beispiel}[unnamedtheorem]{Beispiel}
\newtheorem{folgerung}[unnamedtheorem]{Folgerung}
%\newtheorem{hilfssatz}[unnamedtheorem]{Hilfssatz}
\newtheorem{anwendung}[unnamedtheorem]{Anwendung}
\newtheorem{anwendungen}[unnamedtheorem]{Anwendungen}
\newtheorem*{anwendung*}{Anwendung}
\newtheorem*{beispiel*}{Beispiel}
\newtheorem*{beispiele}{Beispiele}
\newtheorem*{bemerkung}{Bemerkung} 
\newtheorem*{bemerkungen}{Bemerkungen}
\newtheorem*{bezeichnung}{Bezeichnung}
\newtheorem*{eigenschaften}{Eigenschaften}
\newtheorem*{folgerung*}{Folgerung}
\newtheorem*{folgerungen}{Folgerungen}
%\newtheorem*{hilfssatz*}{Hilfssatz}
\newtheorem*{regeln}{Regeln}
\newtheorem*{motivation}{Motivation}
\newtheorem*{erinnerung}{Erinnerung}
\newtheorem*{schreibweise}{Schreibweise}
\newtheorem*{schreibweisen}{Schreibweisen}
\newtheorem*{uebung}{Übung}
\newtheorem*{vereinbarung}{Vereinbarung}

%% Template
\makeatletter%
\DeclareUnicodeCharacter{00A0}{ } \pgfplotsset{compat=1.7} \hypersetup{colorlinks,breaklinks, urlcolor=linkcolor, linkcolor=linkcolor, pdftitle=\@title, pdfauthor=\@author, pdfsubject=\@title, pdfcreator=\@publishers}\DeclareOption*{\PassOptionsToClass{\CurrentOption}{report}} \ProcessOptions \def\baselinestretch{\mystretch} \setlength{\oddsidemargin}{0.125in} \setlength{\evensidemargin}{0.125in} \setlength{\topmargin}{0.5in} \setlength{\textwidth}{6.25in} \setlength{\textheight}{8in} \addtolength{\topmargin}{-\headheight} \addtolength{\topmargin}{-\headsep} \def\pulldownheader{ \addtolength{\topmargin}{\headheight} \addtolength{\topmargin}{\headsep} \addtolength{\textheight}{-\headheight} \addtolength{\textheight}{-\headsep} } \def\pullupfooter{ \addtolength{\textheight}{-\footskip} } \def\ps@headings{\let\@mkboth\markboth \def\@oddfoot{} \def\@evenfoot{} \def\@oddhead{\hbox {}\sl \rightmark \hfil \rm\thepage} \def\chaptermark##1{\markright {\uppercase{\ifnum \c@secnumdepth >\m@ne \@chapapp\ \thechapter. \ \fi ##1}}} \pulldownheader } \def\ps@myheadings{\let\@mkboth\@gobbletwo \def\@oddfoot{} \def\@evenfoot{} \def\sectionmark##1{} \def\subsectionmark##1{}  \def\@evenhead{\rm \thepage\hfil\sl\leftmark\hbox {}} \def\@oddhead{\hbox{}\sl\rightmark \hfil \rm\thepage} \pulldownheader }	\def\chapter{\cleardoublepage  \thispagestyle{plain} \global\@topnum\z@ \@afterindentfalse \secdef\@chapter\@schapter} \def\@makeschapterhead#1{ {\parindent \z@ \raggedright \normalfont \interlinepenalty\@M \Huge \bfseries  #1\par\nobreak \vskip 40\p@ }} \newcommand{\indexsection}{chapter} \patchcmd{\@makechapterhead}{\vspace*{50\p@}}{}{}{}\def\Xint#1{\mathchoice
    {\XXint\displaystyle\textstyle{#1}} {\XXint\textstyle\scriptstyle{#1}} {\XXint\scriptstyle\scriptscriptstyle{#1}} {\XXint\scriptscriptstyle\scriptscriptstyle{#1}} \!\int} \def\XXint#1#2#3{{\setbox0=\hbox{$#1{#2#3}{\int}$} \vcenter{\hbox{$#2#3$}}\kern-.5\wd0}} \def\dashint{\Xint-} \def\Yint#1{\mathchoice {\YYint\displaystyle\textstyle{#1}} {\YYYint\textstyle\scriptscriptstyle{#1}} {}{} \!\int} \def\YYint#1#2#3{{\setbox0=\hbox{$#1{#2#3}{\int}$} \lower1ex\hbox{$#2#3$}\kern-.46\wd0}} \def\YYYint#1#2#3{{\setbox0=\hbox{$#1{#2#3}{\int}$}  \lower0.35ex\hbox{$#2#3$}\kern-.48\wd0}} \def\lowdashint{\Yint-} \def\Zint#1{\mathchoice {\ZZint\displaystyle\textstyle{#1}}{\ZZZint\textstyle\scriptscriptstyle{#1}} {}{} \!\int} \def\ZZint#1#2#3{{\setbox0=\hbox{$#1{#2#3}{\int}$}\raise1.15ex\hbox{$#2#3$}\kern-.57\wd0}} \def\ZZZint#1#2#3{{\setbox0=\hbox{$#1{#2#3}{\int}$} \raise0.85ex\hbox{$#2#3$}\kern-.53\wd0}} \def\highdashint{\Zint-} \DeclareRobustCommand*{\onlyattoc}[1]{} \newcommand*{\activateonlyattoc}{ \DeclareRobustCommand*{\onlyattoc}[1]{##1} } \AtBeginDocument{\addtocontents{toc} {\protect\activateonlyattoc}} 
	% Titlepage
	\def\maketitle{ \begin{titlepage} 
			~\vspace{3cm} 
		\begin{center} {\Huge \@title} \end{center} 
	 		\vspace*{1cm} 
	 	\begin{center} {\large \@author} \end{center} 
	 	\begin{center} \@date \end{center} 
	 		\vspace*{7cm} 
	 	\begin{center} \@publishers \end{center} 
	 		\vfill 
	\end{titlepage} }
\makeatother%

% Indexdatei erstellen
\makeindex 

\begin{document}

\pagenumbering{arabic}
	
% Inhaltsverzeichnis
%\tableofcontents
%\thispagestyle{empty}
  

\chapter*{Evolutionäre Spieltheorie}  
  
\section*{Spiele in Normalform}
Für symmetrische Spiele:
$$ A = \left( a_{ij} \right) \quad i = 1, \dotsc, m_{i}, ~ j = 1, \dotsc, m_{j} $$
d.h.
% game table 
\begin{description}
	\item $N$: Spielermenge $|N| = n$
	\item $\Sigma_{i}$: Menge der reinen Strategien von $i \in N$, $\left| \Sigma_{i} \right| = m_{i}$, $\sigma_{i} \in \mathcal{E}_{i}$.
	\item $S_{i}$: Menge der gemischten Strategien von $i \in N$
		\begin{description}
			\item $S_{i} = \left\{ \right\}$.
			\item $s_{ij} = \mathds{P}(\sigma_{ij})$.
		\end{description}
\end{description}
  
\begin{definition}[Trägermenge] Wir definieren die Trägermenge für jeden Spieler $i \in N$:
	$$ C(S_{i}) = \left\{ \sigma_{ij} \in \Sigma_{i} : s_{ij} > 0 \right\}, $$
	als die Menge der Strategien die mit positiver Wahrscheinlichkeit gespielt werden.
\end{definition}  

\begin{definition}[Beste-Antwort-Menge] Sei
	$$ B_{i}(s_{-i}) = \left\{ \sigma_{j} \in \Sigma_{i} : H(\sigma_{ij}, s_{-i}) = \max_{\sigma_{ik \in \Sigma_{i}}} H(\sigma_{ik}, s_{-i}) \right\} $$ 
	$H$ bezeichne pay-off-Funktion ~\\
	$$ \hat{H}(s_{-i}) \coloneqq \max_{\sigma_{il} \in \Sigma_{i}} H(\sigma_{ik}, s_{-i}) $$
\end{definition}
  
  
\begin{beispiel*}
	$\sigma_{ij} \in B_{i}(S_{-i})$ und $\sigma_{ik} \in B(S_{-i}) \Rightarrow$ alle $s_{i} \in S_{i}$ mit
	$$ C(S_{i}) = \{ \sigma_{ij}, \sigma_{ik} \} $$	
	sind auch beste Antwort, denn
	$$ s_{ij} H(\sigma_{j}, s_{-i}) + s_{ik} H(\sigma_{ik}, s_{-i}) = (s_{ij} + s_{ik}) \hat{H}(s_{-i}) = \hat{H}(s_{-i}). $$
\end{beispiel*}

Sei $s^{*} = \left( s_{1}^{*}, \dotsc, s_{n}^{*} \right)$ ein Nash-Gleichgewicht. Mit $s_{i}^{*} = (s_{i1}^{*}, \dotsc s_{im_{i}}^{*})$ gilt
$$ C(s_{i}^{*}) \subseteq B_{i}(s_{-i}^{*}) $$
 \textit{Hinreichend? Ja! Proposition Slide 36 (AGT Teil 1)}.


\begin{unnamedtheorem}[Grundannahmen der evolutionären Spieltheorie]
	\begin{enumerate}
		\item große Population
		\item Population ist monomorph
		\item random matching
		\item Wettstreit (Spiel) ist statisch und symmetrisch
			$$ \rightarrow \text{ symmetrisches Spiel in Normalform mit zwei Spielern}. $$
		\item Auszahlung entspricht der \enquote{biologischen Fitness} ($\phi$ Anzahl Nachkommen)
		\item Reproduktion ist asexuell und die von den Eltern gewählt Strategie wird unverändert an die Nachkommen vererbt (nur Selektion, keine Mutation).
	\end{enumerate}
\end{unnamedtheorem} 
 

\begin{unnamedtheorem}[Symmetrisches 2-Personenspiel in Normalform]
	Spieler müssen nicht unterschieden werden $\Rightarrow$ Strategieraum:
	$$ S = \{ s\in \R^{m} : \sum_{i=1}^{m} s_{i} = 1, s_{i} \geq 0, i = 1, \dotsc, m \} $$
\end{unnamedtheorem}
  
 
\begin{definition}[Evolutionär stabile Strategie, ESS]
	Eine Strategie $p \in S$ hei{\ss}t evolutionär stabil, wenn
	\begin{enumerate}
		\item $H(p,p) \geq H(q, p)$ für alle $q \in S$ (Gleichgewichtsbedingung)
		\item Für alle $q \in S \setminus \{ p \}$ mit $H(q, p) = H(p, p)$ gilt: $H(p, q) > H(q, q)$ (Stabilitätsbedingung)
	\end{enumerate}
\end{definition}  


\begin{unnamedtheorem}[Eigenschaften von evolutionär stabile Strategie]
	\begin{itemize}
		\item Ist $p \in S$ eine evolutionär stabile Strategie, dann bildet $(p, p)$ ein symmetrisches Nash-Gleichgewicht
		\item Jede $2 \times 2$-Matrix $A = \begin{pmatrix}
			a_{11} & a_{12} \\ a_{21} & a_{22}
		\end{pmatrix}$ mit $H(p,p) = p'Ap$ sodass $a_{11} \neq a_{21}$ und / oder $a_{12} = a_{22}$, besitzt eine ESS
		Ist $(p,p)$ ein striktes NGG, dann ist $p$ eine ESS. Im strikten NGG $(p, p)$ gilt $C(p) = B(p)$. Ein striktes NGG ist immer ein Gleichgewicht in reinen Strategien. Beispiel:
   \begin{game}{2}{2}[~][]
   	    &  ~      &  ~     \\
   	 ~  &    $3, 3$      & $2, 0$  \\
   	  	&  $0, 2$ & $4, 4$\\
   \end{game}
	\item Im Normalformspielen mit $m \times m$-Matrizen $a$ mit $m \geq 3$ existieren entweder endlich viele ESS keine.
	\end{itemize}
\end{unnamedtheorem}
  
\subsection*{Aufgaben}

\subsubsection*{2) b)}  
      $$ \begin{game}{2}{2}[~][~]
   	    &  $\sigma_{1}$      &  $\sigma_{2}$     \\
   	 $\sigma_{1}$  &    0, 0      & 0, 0  \\
   	 $\sigma_{2}$ 	&  0, 0 & 1, 1 \\
   \end{game}$$ $\Rightarrow A = \begin{pmatrix} 0 & 0 \\ 0 & 1 \end{pmatrix} \text{ (Gegenbeispiel)}$
   $$ \sigma^{*} = (\sigma_1, \sigma_1) , \sigma^{**} = (\sigma_2, \sigma_2) $$
  
  \begin{enumerate}[label=\alph*\upshape)]
  	\item Angenommen $p \in S$ ist ESS  und wird von $q \in S$ schwach dominiert
  		$$ \Rightarrow H(q, z) \geq H(p, z) \quad \forall z \in S $$
  		\begin{align*}
  			\Rightarrow & H(p,p) = H(q, p) \quad \text{ Bedingung 1: ok} \\
  			\Rightarrow & H(p,q) \leq H(q, q) \quad \text{ Bedingung 2: verletzt} \\
  		\end{align*}
  	\item s.o.
  	\item klar!
  \end{enumerate}
   
\subsubsection*{1) Hawp-Dove-Game / Falke-Taube-Spiel}   
	$$
		\begin{game}{2}{2}[~][~]
   	   			&  $F$      &  $T$     \\
   	 $F$  &    $\frac{v-c}{2}$, $\frac{v-c}{2}$      & $v, 0$  \\
   	 $T$ 	&  $v, 0$ & $\frac{v}{2}$, $\frac{v}{2}$ \\
   \end{game}$$
   $A = \begin{pmatrix}
   	\frac{v-c}{2} & v \\ 0 & \frac{v}{2}
   \end{pmatrix}$, $c > v > 0$
   \begin{itemize}
   	\item es existiert keine dominante Strategie
   	\item es existiert kein symmetrisch Nash-Gleichgewicht in reinen Strategien
   	\item $(F, T)$, $(T, F)$ sind strikte Nash-Gleichgewichte
   \end{itemize}
   Interpretation: Recource $v$, Tauben teilen friedlich, Falken vertielgt Zaube, Falken kämpfen $\rightarrow$ neg, outcome für beide. ~\\
   $p = (p_F, p_T)$
  $$	H(F, p) \overset{!}{=} H(T, p) \overset{!}{=} H(p,p) $$
   $$	\begin{rcases} H(F, p) & = p_{F} \frac{v-c}{2} + p_T v \\ H(T, p) & = p_F 0 + p_T \frac{v}{2} \end{rcases} \xrightarrow[]{p_F + p_T =1} p_F = \frac{v}{c}, p_T = 1 - \frac{v}{c} $$
   ist das einzige symmetrisch Nas-Gleichgewicht, kein triviales Spiel $\Rightarrow \exists ESS$
   $$ \Rightarrow \left( \frac{v}{2}, 1 - \frac{v}{2} \right) \text{ ist ESS} $$
   oder man rechnet nach $H(F,p) = H(t, p) = H(p,p)$ ~\\
   $\Rightarrow z.z. H(p, F) > H(F, F)$, $H(p, T) > H(T, T)$

\subsubsection*{3)}
$$A = \begin{pmatrix} 1 & 1 & 0 \\ 1& 1 & 1 \\ 0 & 1 & 1 \end{pmatrix}$$   
\begin{enumerate}
	\item Nash-Gleichgewicht in reinen Strategien:
		$$ (x,x), ~(x,y), ~(y,x), ~(y,z), ~(z, y),  ~(z,z) $$
		Trivial: $C(A) = \{ A \}$, $A \in \{ x,y,z\}$
		$$ B(x) = \{x,y\}, ~B(y) = \{x, y,z\}, ~B(z) = \{ y, z \} $$
		$$ \Rightarrow C(\cdot) \underset{\not-}{\subset} B(\cdot) $$
		Nash-Gleichgewicht in gemischten Strategien (nur sym.)
		$$ S^{*} = \{ (s_x, s_y, 0) : s_x \in (0, 1), s_y = 1 - s_x \} \quad C(S^{*}) = \{x, y\} $$
		$$ S^{**} = \{ (0, s_y, s_z) : s_y \in (0, 1), s_z = 1 - s_y \} \quad C(S^{**}) = \{y, z\} $$
		$B(S^{*}) = \{ x, y \}$, $B(S^{**}) = \{y, z \}$
	\item Angenommen $p \in S$ mit $p_x \in [0, 1]$ und $p_y = 1 - p_x$ ist ein ESS
		\begin{description}
			\item Bedingung 1: $\checkmark$
			\item Bedingung 2: $H(x, p) = H(p, p) = 1$ mit $p_x < 1$
				$$ \Rightarrow H(p, x) > H(x, x) \text{ Widerspruch!} $$
				analog in anderen Fällen $\Rightarrow $ ESS existiert nicht.
		\end{description}
\end{enumerate}
   
\begin{unnamedtheorem}[Allgemein gilt]
	Ist $p$ ESS $\Rightarrow \neg \exists \sigma \in C(p)$ mit $\sigma \in C(S^{*})$ für $s^{*} \neq q$ ist Nash-Gleichgewicht
\end{unnamedtheorem} 

$\Rightarrow \#$ ESS $\leq \left| \Sigma \right|$ - Gleichheit nur, fall es kein ESS in gemischten Strategien gibt.  
   
\end{document}