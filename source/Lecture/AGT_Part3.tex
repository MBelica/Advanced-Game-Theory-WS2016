\documentclass[12pt]{extreport} % Schriftgröße: 8pt, 9pt, 10pt, 11pt, 12pt, 14pt, 17pt oder 20pt

%% Packages
\usepackage{scrextend}
\usepackage{amssymb}
\usepackage{amsthm}
\usepackage{booktabs}
\usepackage{chngcntr}
\usepackage{cmap}
\usepackage{color}
\usepackage{enumitem}
\usepackage{float}
\usepackage{hyperref}
\usepackage{ulem}
\usepackage{lmodern}
\usepackage{makeidx}
\usepackage{dsfont}
\usepackage{mathtools}
\usepackage{xpatch}
\usepackage{pgfplots}
\pgfplotsset{compat=1.7}
\usetikzlibrary{calc}	
\usetikzlibrary{matrix}	
\usepackage{minibox}
\usepackage{xcolor}
\usepackage{sgame} % Game theory packages
\usepackage{subfig} % Manipulation and reference of small or sub figures and tables

\usepgfplotslibrary{fillbetween}
\usetikzlibrary{patterns}
\usetikzlibrary{decorations.markings}
\usetikzlibrary{calc, intersections}
\usetikzlibrary{trees, calc} % For extensive form games
\pgfplotsset{compat=1.7}
\usetikzlibrary{calc}	
\usetikzlibrary{matrix}	


% Language Setup (Deutsch)
\usepackage[utf8]{inputenc} 
\usepackage[T1]{fontenc} 
\usepackage[english]{babel}

\usepackage{csquotes}
% Options
\makeatletter%%  
  % Linkfarbe, {0,0.35,0.35} für Türkis, {0,0,0} für Schwarz, {1,0,0} für Rot, {0,0,0.85} für Blau
  \definecolor{linkcolor}{rgb}{0,0.35,0.35}
  % Zeilenabstand für bessere Leserlichkeit
  \def\mystretch{1.2} 
  % Publisher definieren
  \newcommand\publishers[1]{\newcommand\@publishers{#1}} 
  % Enumerate im 1. Level: \alph für a), b), ...
  \renewcommand{\labelenumi}{\alph{enumi})} 
  % Enumerate im 2. Level: \roman für (i), (ii), ...
  \renewcommand{\labelenumii}{(\roman{enumii})}
  % Zeileneinrückung am Anfang des Absatzes
  \setlength{\parindent}{0pt} 
  % Verweise auf Enumerate, z.B.: 3.2 a)
  \setlist[enumerate,1]{ref={\thesatz ~ \alph*)}}
  % Für das Proof-Environment: 'Beweis:' anstatt 'Beweis.'
  \xpatchcmd{\proof}{\@addpunct{.}}{\@addpunct{:}}{}{} 
  % Nummerierung der Bilder, z.B.: Abbildung 4.1
  \@ifundefined{thechapter}{}{\def\thefigure{\thechapter.\arabic{figure}}} 
  % Chapter-Nummerierung beginnen bei (-1):
  \setcounter{chapter}{14}
\makeatother%

% Meta Setup (Für Titelblatt und Metadaten im PDF)
\title{Höhere Mathematik II}
\author{G. Herzog, Ch. Schmoeger}
\date{Sommersemester 2017}
\publishers{Karlsruher Institut für Technologie}

%% Math. Definitionen
\newcommand{\C}{\mathbb{C}}
\newcommand{\N}{\mathbb{N}}
\newcommand{\Q}{\mathbb{Q}}
\newcommand{\R}{\mathbb{R}}
\newcommand{\Z}{\mathbb{Z}}

%% Theorems (unnamedtheorem = Theorem ohne Namen)
\newtheoremstyle{named}{}{}{\normalfont}{}{\bfseries}{:}{0.25em}{#2 \thmnote{#3}}
\newtheoremstyle{nnamed}{}{}{\normalfont}{}{\bfseries}{:}{0.25em}{\thmnote{#3}}
\newtheoremstyle{itshape}{}{}{\itshape}{}{\bfseries}{:}{ }{}
\newtheoremstyle{normal}{}{}{\normalfont}{}{\bfseries}{:}{ }{}
\renewcommand*{\qed}{\hfill\ensuremath{\square}}

\theoremstyle{named}
\newtheorem{unnamedtheorem}{Theorem} \counterwithin{unnamedtheorem}{chapter}
\theoremstyle{nnamed}
\newtheorem*{unnamedtheorem*}{Theorem} 

\theoremstyle{itshape}
\newtheorem{satz}[unnamedtheorem]{Satz} 
\newtheorem*{definition}{Definition}
\newtheorem{hilfssatz}[unnamedtheorem]{Hilfssatz}
\newtheorem*{hilfssatz*}{Hilfssatz}

\theoremstyle{normal}
\newtheorem{beispiel}[unnamedtheorem]{Beispiel}
\newtheorem{folgerung}[unnamedtheorem]{Folgerung}
%\newtheorem{hilfssatz}[unnamedtheorem]{Hilfssatz}
\newtheorem{anwendung}[unnamedtheorem]{Anwendung}
\newtheorem{anwendungen}[unnamedtheorem]{Anwendungen}
\newtheorem*{anwendung*}{Anwendung}
\newtheorem*{beispiel*}{Beispiel}
\newtheorem*{beispiele}{Beispiele}
\newtheorem*{bemerkung}{Bemerkung} 
\newtheorem*{bemerkungen}{Bemerkungen}
\newtheorem*{bezeichnung}{Bezeichnung}
\newtheorem*{eigenschaften}{Eigenschaften}
\newtheorem*{folgerung*}{Folgerung}
\newtheorem*{folgerungen}{Folgerungen}
\newtheorem*{proposition*}{Proposition}
%\newtheorem*{hilfssatz*}{Hilfssatz}
\newtheorem*{regeln}{Regeln}
\newtheorem*{motivation}{Motivation}
\newtheorem*{erinnerung}{Erinnerung}
\newtheorem*{schreibweise}{Schreibweise}
\newtheorem*{schreibweisen}{Schreibweisen}
\newtheorem*{uebung}{Übung}
\newtheorem*{vereinbarung}{Vereinbarung}

%% Template
\makeatletter%
\DeclareUnicodeCharacter{00A0}{ } \pgfplotsset{compat=1.7} \hypersetup{colorlinks,breaklinks, urlcolor=linkcolor, linkcolor=linkcolor, pdftitle=\@title, pdfauthor=\@author, pdfsubject=\@title, pdfcreator=\@publishers}\DeclareOption*{\PassOptionsToClass{\CurrentOption}{report}} \ProcessOptions \def\baselinestretch{\mystretch} \setlength{\oddsidemargin}{0.125in} \setlength{\evensidemargin}{0.125in} \setlength{\topmargin}{0.5in} \setlength{\textwidth}{6.25in} \setlength{\textheight}{8in} \addtolength{\topmargin}{-\headheight} \addtolength{\topmargin}{-\headsep} \def\pulldownheader{ \addtolength{\topmargin}{\headheight} \addtolength{\topmargin}{\headsep} \addtolength{\textheight}{-\headheight} \addtolength{\textheight}{-\headsep} } \def\pullupfooter{ \addtolength{\textheight}{-\footskip} } \def\ps@headings{\let\@mkboth\markboth \def\@oddfoot{} \def\@evenfoot{} \def\@oddhead{\hbox {}\sl \rightmark \hfil \rm\thepage} \def\chaptermark##1{\markright {\uppercase{\ifnum \c@secnumdepth >\m@ne \@chapapp\ \thechapter. \ \fi ##1}}} \pulldownheader } \def\ps@myheadings{\let\@mkboth\@gobbletwo \def\@oddfoot{} \def\@evenfoot{} \def\sectionmark##1{} \def\subsectionmark##1{}  \def\@evenhead{\rm \thepage\hfil\sl\leftmark\hbox {}} \def\@oddhead{\hbox{}\sl\rightmark \hfil \rm\thepage} \pulldownheader }	\def\chapter{\cleardoublepage  \thispagestyle{plain} \global\@topnum\z@ \@afterindentfalse \secdef\@chapter\@schapter} \def\@makeschapterhead#1{ {\parindent \z@ \raggedright \normalfont \interlinepenalty\@M \Huge \bfseries  #1\par\nobreak \vskip 40\p@ }} \newcommand{\indexsection}{chapter} \patchcmd{\@makechapterhead}{\vspace*{50\p@}}{}{}{}\def\Xint#1{\mathchoice
    {\XXint\displaystyle\textstyle{#1}} {\XXint\textstyle\scriptstyle{#1}} {\XXint\scriptstyle\scriptscriptstyle{#1}} {\XXint\scriptscriptstyle\scriptscriptstyle{#1}} \!\int} \def\XXint#1#2#3{{\setbox0=\hbox{$#1{#2#3}{\int}$} \vcenter{\hbox{$#2#3$}}\kern-.5\wd0}} \def\dashint{\Xint-} \def\Yint#1{\mathchoice {\YYint\displaystyle\textstyle{#1}} {\YYYint\textstyle\scriptscriptstyle{#1}} {}{} \!\int} \def\YYint#1#2#3{{\setbox0=\hbox{$#1{#2#3}{\int}$} \lower1ex\hbox{$#2#3$}\kern-.46\wd0}} \def\YYYint#1#2#3{{\setbox0=\hbox{$#1{#2#3}{\int}$}  \lower0.35ex\hbox{$#2#3$}\kern-.48\wd0}} \def\lowdashint{\Yint-} \def\Zint#1{\mathchoice {\ZZint\displaystyle\textstyle{#1}}{\ZZZint\textstyle\scriptscriptstyle{#1}} {}{} \!\int} \def\ZZint#1#2#3{{\setbox0=\hbox{$#1{#2#3}{\int}$}\raise1.15ex\hbox{$#2#3$}\kern-.57\wd0}} \def\ZZZint#1#2#3{{\setbox0=\hbox{$#1{#2#3}{\int}$} \raise0.85ex\hbox{$#2#3$}\kern-.53\wd0}} \def\highdashint{\Zint-} \DeclareRobustCommand*{\onlyattoc}[1]{} \newcommand*{\activateonlyattoc}{ \DeclareRobustCommand*{\onlyattoc}[1]{##1} } \AtBeginDocument{\addtocontents{toc} {\protect\activateonlyattoc}} 
	% Titlepage
	\def\maketitle{ \begin{titlepage} 
			~\vspace{3cm} 
		\begin{center} {\Huge \@title} \end{center} 
	 		\vspace*{1cm} 
	 	\begin{center} {\large \@author} \end{center} 
	 	\begin{center} \@date \end{center} 
	 		\vspace*{7cm} 
	 	\begin{center} \@publishers \end{center} 
	 		\vfill 
	\end{titlepage} }
\makeatother%

% Indexdatei erstellen
\makeindex 

\begin{document}

\pagenumbering{arabic}
	
% Inhaltsverzeichnis
%\tableofcontents
%\thispagestyle{empty}
  

\chapter*{Mechanism and Market Design}  
  
\section*{Markets: Trade}

Third part of the lecture held by Prof. Szech. By bilateral trades you often find a compromise in a trade that can be profitable for both parties (see movie clip).

\subsection*{The Importance of Bilateral Trade}

For the game theoretical analysis we need some basics: ~\\

\textbf{Setting}:
 
\begin{itemize}
	\item 1 Buyer ($B$), 1 Seller ($S$)
	\item $S$ ca produce an object of cost $c$
	\item $B$ likes the object $v$ much
	\item $v$ is private info to $B$, common info is only $\tilde{v} \sim U[0, 1]$
	\item $c$ is private information to $S$, and again we assume $\tilde{c} \sim U[0, 1]$ for $B$.
	\item Trade at price $p$ will lead to profits:
		\begin{itemize}
			\item $\pi_{S} = p - c$ for $S$
			\item $\pi_{B} = v - p$ for $B$
		\end{itemize}
\end{itemize}
~\newpage
For review, the uniform distribution: $U[0, 1]$:

\begin{figure*}[ht]
    \begin{center}
		\begin{tikzpicture}[scale=0.8,
  					declare function={ funcA(\x)= 
  						(\x<=1) * (1);
  					},
				]
			\begin{axis}[
  						 axis x line=middle, 
  						 axis y line=middle,
  						 xmin=-0.05, xmax=1.25, 
  						 xtick={0.5,1},
   						 ymin=-0.05, ymax=1.25, 
  						 ytick={0.51}, 
						]
				\addplot[blue, solid, domain=0:1,name path=A]{funcA(x)};
				\addplot +[blue, dashed, mark=none] coordinates {(1, 0) (1, 1)};	
				\node[anchor=south] at (axis cs:1,1) {\footnotesize $U[0, 1]$};
				\node[anchor=south] at (axis cs:-0.03,1.1) {\footnotesize $f$};
				\node[anchor=south] at (axis cs:1.2,-0.01) {\footnotesize $x$};
			\end{axis}
		\end{tikzpicture}
	  \end{center}
\end{figure*}
  
Mechanism - Double Auction (V. Smith):
\begin{enumerate}
	\item $B$ and $S$ will simultaneously state prices $p_{S}$, $p_{B}$ ~\\ ~\\
		  \tikz[baseline=-0.5ex]{  
			\draw(0,0)--(10,0);
				\foreach \x/\xtext in {0/$0$,2/$$,4/$p_{B}$,6/{},8/$p_{S}$,10/$1$}
      				\draw(\x,3pt)--(\x,-3pt) node[below] {\xtext};
    				\draw[decorate,decoration={brace},yshift=1ex]  (0,0) -- node[above=0.4ex] {\footnotesize can not be realised since $p_{S} > p_{B}$}  (10,0); } ~\\
    	  \tikz[baseline=-0.5ex]{  
			\draw(0,0)--(10,0);
   				\foreach \x/\xtext in {0/$0$,2/$$,4/$p_{S}$,6/{},8/$p_{B}$,10/$1$}
      				\draw(\x,3pt)--(\x,-3pt) node[below] {\xtext};
    				\draw[decorate,decoration={brace},yshift=1ex]  (0,0) -- node[above=0.4ex] {\footnotesize can be realised since $p_{B} > p_{S}$ and a trade occurs}  (10,0); }
	\item Trade occurs if and only if $p_{B} \geq p_{S}$. Trading price $p = \frac{p_{B} + p_{S}}{2}$.
\end{enumerate} 

Fixed Price Equilibrium: Q: Can we find a BNE such that $p = \frac{1}{2}$ is achieved? A Bayes-Nash-Equilibrium that satisfies this condition if for example:
$$ p_{S}(c) = \begin{cases} \frac{1}{2} & \text{if } 0 \leq \frac{1}{2} \\ c & \text{if } c > \frac{1}{2} \end{cases}, \quad p_{B}(v) = \begin{cases} \frac{1}{2} & \text{if } v \geq \frac{1}{2} \\ v & \text{if } v < \frac{1}{2} \end{cases}  $$
hence, this is a Fixed Price Equilibrium with fixed price $p = p_{S} = p_{B} = \frac{1}{2}$.  ~\\
  
\begin{proposition*}
	For $x \in (0, 1)$, the following strategies specify a fixed price (Bayesian) Nash-Equilibrium.
	$$ p_{B}(v) = \begin{cases} x, & \text{if } v \geq x \\ 0, & \text{if } v < x \end{cases}, \quad p_{B}(c) = \begin{cases} x, & \text{if } c \leq x \\ 1, & \text{if } c > x \end{cases} \quad  $$
	
	\begin{proof} ~\\ ~\\
	  Considering the buyer:
		\begin{enumerate}
			\item Case 1: $v < x$:
				\begin{center}
					\begin{tabular}{l|l}
  						action & profit \\
  					  	  \hline
  						$p_{B} \in [0, x)$ & $\pi = 0$ (no trade) \\
 						$p_{B} \in [x, 1]$ & $\pi < 0$ (trade can happen)
					\end{tabular}
				\end{center}
			\item Case 2: $v \geq x$: 
				\begin{center}
					\begin{tabular}{l|l}
  						action & profit \\
  					  	  \hline
  						$p_{B} = [0, x)$ & $\pi = 0$ (no trade) \\
  						$p_{B} = x$      & $\pi \geq 0$ (trade can happen) \\
 						$p_{B} = (x, 1]$ &
					\end{tabular}
				\end{center}
		\end{enumerate}
		
		Considering the seller: (task 1) ~\\ ~\\
		
		depict (task 2):
		\begin{enumerate}[label=\alph*\upshape)]
			\item a
			\item  When will trades occur according to the potential Equilibrium specified?
		\end{enumerate}
	\end{proof}
\end{proposition*}
  
  
\newpage 
  
\section*{Exam}  
  
\begin{enumerate}
	\item Theorie question, tasks in lecture are more or less the essentials - calculate and understand theory behind
	\item Read paper, state in a few words what the paper is about - what do you think about the paper, extend improve or criticise the paper - what do you like, don't like about the paper or what is important 
\end{enumerate}  
  
\end{document}