\documentclass[12pt]{extreport} % Schriftgröße: 8pt, 9pt, 10pt, 11pt, 12pt, 14pt, 17pt oder 20pt

% Language Setup (Deutsch)
\usepackage[utf8]{inputenc} 
\usepackage[english]{babel}

%% Packages
\usepackage{scrextend}
\usepackage{amssymb}
\usepackage{amsthm}
\usepackage{amsmath}
\usepackage[inline]{enumitem}
\usepackage{changes}
\usepackage{chngcntr}
\usepackage{cmap}
\usepackage{color}
\usepackage{csquotes}
\usepackage{float}
\usepackage{hyperref}
\usepackage{footnote}

\usepackage{lmodern}
\usepackage{makeidx}
\usepackage{mathtools} 
\usepackage{xpatch}
\usepackage{pgfplots}
\usepackage{stmaryrd}
\usepackage{pbox}
\usepackage{apptools}
\usepackage{booktabs}
\usepackage{dsfont}
\usepackage{graphicx}
\usepackage{mathrsfs}
\usepackage[square,numbers]{natbib}
\usepackage{nicefrac}
\usepackage{pgf}
\usepackage{pgfplots}
\usepackage{tikz}
\usepackage{tocloft}
\usepackage{url}
\usepackage{xpatch}
\usepackage{microtype}
\usepackage{pgfplots}
\usepackage{minibox}
\usepackage{xcolor}
\usepackage{sgame} % Game theory packages
\usepackage{subfig} % Manipulation and reference of small or sub figures and tables

\makesavenoteenv{tabular}
\usepgfplotslibrary{fillbetween}
\usetikzlibrary{patterns}
\usetikzlibrary{decorations.markings}
\usetikzlibrary{calc, intersections}
\usetikzlibrary{trees, calc} % For extensive form games
\pgfplotsset{compat=1.7}
\usetikzlibrary{calc}	
\usetikzlibrary{matrix}	

\usepgfplotslibrary{fillbetween}
\usetikzlibrary{patterns}
\usetikzlibrary{decorations.markings}
\usetikzlibrary{calc, intersections}
\usetikzlibrary{trees, calc} % For extensive form games
\pgfplotsset{compat=1.7}
\usetikzlibrary{calc}	
\usetikzlibrary{matrix}	

% Options
\makeatletter%%  
  % Linkfarbe, {0,0.35,0.35} für Türkis, {0,0,0} für Schwarz 
  \definecolor{linkcolor}{rgb}{0,0.35,0.35}
  % Zeilenabstand für bessere Leserlichkeit
  \def\mystretch{1.5} 
  % Publisher definieren
  \newcommand\publishers[1]{\newcommand\@publishers{#1}} 
  % Enumerate im 1. Level: \alph für a), b), ...
  \renewcommand{\labelenumi}{\alph{enumi})} 
  % Enumerate im 2. Level: \roman für (i), (ii), ...
  \renewcommand{\labelenumii}{(\roman{enumii})}
  % Zeileneinrückung am Anfang des Absatzes
  \setlength{\parindent}{0pt} 
  % Verweise auf Enumerate, z.B.: 3.2 a)
  \setlist[enumerate,1]{ref={\thesatz ~ \alph*)}}
  % Für das Proof-Environment: 'Beweis:' anstatt 'Beweis.'
  \xpatchcmd{\proof}{\@addpunct{.}}{\@addpunct{:}}{}{} 
  % Nummerierung der Bilder, z.B.: Abbildung 4.1
  \@ifundefined{thechapter}{}{\def\thefigure{\thechapter.\arabic{figure}}} 
\makeatother%

% Meta Setup (Für Titelblatt und Metadaten im PDF)
\title{Advanced Game Thoerie}
\author{Prof. Dr. Clemens Puppe}
\date{(inoffizielles Skript) ~\vspace{0.2cm} \\ Wintersemester 2016/17}
\publishers{Karlsruher Institut für Technologie}

%% Math. Definitions
\newcommand{\C}{\mathbb{C}}
\newcommand{\N}{\mathbb{N}}
\newcommand{\Q}{\mathbb{Q}}
\newcommand{\R}{\mathbb{R}}
\newcommand{\Z}{\mathbb{Z}}

%% Theorems (unnamedtheorem = Theorem ohne Namen)
\newtheoremstyle{named}{}{}{\normalfont}{}{\bfseries}{:}{0.25em}{#2 \thmnote{#3}}
\newtheoremstyle{itshape}{}{}{\itshape}{}{\bfseries}{:}{ }{}
\newtheoremstyle{normal}{}{}{\normalfont}{}{\bfseries}{:}{ }{}
\renewcommand*{\qed}{\hfill\ensuremath{\square}}

\theoremstyle{named}
\newtheorem{unnamedtheorem}{Theorem} \counterwithin{unnamedtheorem}{chapter}
\newtheorem*{unnamedtheorem*}{Theorem} 

\theoremstyle{itshape}
\newtheorem{theorem}[unnamedtheorem]{Theorem}	
\newtheorem{satz}[unnamedtheorem]{Satz}	
\newtheorem*{definition}{Definition}

\theoremstyle{normal}
\newtheorem{beispiel}[unnamedtheorem]{Beispiel}
\newtheorem{example}[unnamedtheorem]{Example}
\newtheorem{folgerung}[unnamedtheorem]{Folgerung}
\newtheorem{hilfssatz}[unnamedtheorem]{Hilfssatz}
\newtheorem{proposition}[unnamedtheorem]{Proposition}
\newtheorem{anwendung}[unnamedtheorem]{Anwendung}
\newtheorem{anwendungen}[unnamedtheorem]{Anwendungen}
\newtheorem*{beispiel*}{Beispiel}
\newtheorem*{example*}{Example}
\newtheorem*{beispiele}{Beispiele}
\newtheorem*{examples}{Examples}
\newtheorem*{bemerkung}{Bemerkung} 
\newtheorem*{comment*}{Comment} 
\newtheorem*{bemerkungen}{Bemerkungen}
\newtheorem*{bezeichnung}{Bezeichnung}
\newtheorem*{eigenschaften}{Eigenschaften}
\newtheorem*{erinnerung}{Erinnerung}
\newtheorem*{folgerung*}{Folgerung}
\newtheorem*{folgerungen}{Folgerungen}
\newtheorem*{hilfssatz*}{Hilfssatz}
\newtheorem*{regeln}{Regeln}
\newtheorem*{schreibweise}{Schreibweise}
\newtheorem*{schreibweisen}{Schreibweisen}
\newtheorem*{uebung}{Übung}
\newtheorem*{vereinbarung}{Vereinbarung}

%% Template
\makeatletter%
\DeclareUnicodeCharacter{00A0}{ } \pgfplotsset{compat=1.7} \hypersetup{colorlinks,breaklinks, urlcolor=linkcolor, linkcolor=linkcolor, pdftitle=\@title, pdfauthor=\@author, pdfsubject=\@title, pdfcreator=\@publishers}\DeclareOption*{\PassOptionsToClass{\CurrentOption}{report}} \ProcessOptions \def\baselinestretch{\mystretch} \setlength{\oddsidemargin}{0.125in} \setlength{\evensidemargin}{0.125in} \setlength{\topmargin}{0.5in} \setlength{\textwidth}{6.25in} \setlength{\textheight}{8in} \addtolength{\topmargin}{-\headheight} \addtolength{\topmargin}{-\headsep} \def\pulldownheader{ \addtolength{\topmargin}{\headheight} \addtolength{\topmargin}{\headsep} \addtolength{\textheight}{-\headheight} \addtolength{\textheight}{-\headsep} } \def\pullupfooter{ \addtolength{\textheight}{-\footskip} } \def\ps@headings{\let\@mkboth\markboth \def\@oddfoot{} \def\@evenfoot{} \def\@oddhead{\hbox {}\sl \rightmark \hfil \rm\thepage} \def\chaptermark##1{\markright {\uppercase{\ifnum \c@secnumdepth >\m@ne \@chapapp\ \thechapter. \ \fi ##1}}} \pulldownheader } \def\ps@myheadings{\let\@mkboth\@gobbletwo \def\@oddfoot{} \def\@evenfoot{} \def\sectionmark##1{} \def\subsectionmark##1{}  \def\@evenhead{\rm \thepage\hfil\sl\leftmark\hbox {}} \def\@oddhead{\hbox{}\sl\rightmark \hfil \rm\thepage} \pulldownheader }	\def\chapter{\cleardoublepage  \thispagestyle{plain} \global\@topnum\z@ \@afterindentfalse \secdef\@chapter\@schapter} \def\@makeschapterhead#1{ {\parindent \z@ \raggedright \normalfont \interlinepenalty\@M \Huge \bfseries  #1\par\nobreak \vskip 40\p@ }} \newcommand{\indexsection}{chapter} \patchcmd{\@makechapterhead}{\vspace*{50\p@}}{}{}{}
	% Titlepage
	\def\maketitle{ \begin{titlepage} 
			~\vspace{3cm} 
		\begin{center} {\Huge \@title} \end{center} 
	 		\vspace*{1cm} 
	 	\begin{center} {\large \@author} \end{center} 
	 	\begin{center} \@date \end{center} 
	 		\vspace*{7cm} 
	 	\begin{center} \@publishers \end{center} 
	 		\vfill 
	\end{titlepage} }
\makeatother%

% Indexdatei erstellen
\makeindex 

\begin{document}

\pagenumbering{Alph}
\begin{titlepage}
	\maketitle
	\thispagestyle{empty}
\end{titlepage}
\pagenumbering{arabic}
	
% Inhaltsverzeichnis
\tableofcontents
\thispagestyle{empty} 
  
% Skript - Anfang 
\chapter{Noncooperative Games}

\section{Basic Elements of Noncooperative Games}


\begin{definition}
	A \textbf{game} is a formal representation of a situation in which a number of individuals interact in a setting of strategic interdependence.
	
	\begin{itemize}
		\item The players: Who is involved?
		\item The rules: Who moves when? What do they know when they move? What can they do?
		\item The outcomes: For each possible set of actions by the players, what is the outcome of the game?
		\item The payoffs: What are the players' preferences over the possible outcomes?
	\end{itemize} 
\end{definition} 


\begin{example}[of simultaneous move games]  ~\
	\begin{enumerate}
		\item Matching Pennies
			\begin{figure}[h!] \centering
  				\begin{game}{2}{2}[Player $1$][Player $2$]
   	    			   	 	&	  Heads    &  Tails   \\
   	 				Heads   &    $-1, 1$   & $1, -1$  \\
   	 				Tails   &    $1, -1$   & $-1, 1$  \\
   				\end{game}
			\end{figure}
		\item Meeting in New York ~\\
			\begin{figure}[h!] \centering
  				\begin{game}{2}{2}[Player $1$][Player $2$]
   	    			   	 		&	Empire State    &  Grand Central   \\
   	 				Empire State   &    $100, 100$   & $0, 0$  \\
   	 				Grand Central   &    $0,0$   & $100, 100$  \\
   				\end{game}
			\end{figure}
		\item Examples of (simple) dynamic games
			 \begin{figure}[h!]
 	\centering
	\caption*{Prisoner's Dilemma in Extensive-form \label{PDExtensive}}
    \begin{tikzpicture}[thin,
      level 1/.style={sibling distance=40mm},
      level 2/.style={sibling distance=25mm},
      level 3/.style={sibling distance=15mm},
      every circle node/.style={minimum size=1.5mm,inner sep=0mm}]
      
      \node[circle,draw,label=above:$P_1$] (root) {}
        child { node [circle,fill,label=above:$P_2$] {}
          child { 
            node {$($-$1,1)$}
            edge from parent
              node[left] {$H$}}
          child { 
            node {$(1,$-$1)$}
            edge from parent
              node[right] {$T$}}
          edge from parent
            node[left] {$H$}}
        child { node [circle,fill,label=above:$P_{2}$] {}
          child { 
            node {$(1,$-$1)$}
              edge from parent
                node[left] {$H$}}
          child { 
            node {$($-$1,1)$}
              edge from parent
                node[right] {$T$}}
           edge from parent
             node[right] {$T$}};
    \end{tikzpicture}
  \end{figure}
		\item Matching Pennies Version C
	% Node styles
	\tikzset{
	% Two node styles for game trees: solid and hollow
	solid node/.style={circle,draw,inner sep=1.5,fill=black},
	hollow node/.style={circle,draw,inner sep=1.5}
	} 

\begin{figure}[h!]
	\centering
	\begin{tikzpicture}[scale=1.5,font=\footnotesize]
	% Specify spacing for each level of the tree
	\tikzstyle{level 1}=[level distance=15mm,sibling distance=40mm]
	\tikzstyle{level 2}=[level distance=15mm,sibling distance=22mm]
	\tikzstyle{level 3}=[level distance=15mm,sibling distance=15mm]
	% The Tree
	\node(0)[hollow node,label=above:{$P1$}]{}
	child{node(1)[solid node]{}
		child{node[label=below:{$($-$1,1)$}]{} edge from parent node[left]{$H$}}
		child{node(3)[label=below:{$(1,$-$1)$}]{} edge from parent node[right]{$T$}}
		edge from parent node[left,xshift=-3]{$H$}
	}
	child{node(2)[solid node]{}
		child{node[label=below:{$(1,$-$1)$}]{} edge from parent node[left]{$H$}}
		child{node[label=below:{$($-$1,1)$}]{} edge from parent node[right]{$T$}}
		edge from parent node[right,xshift=3]{$T$}
	};
	% information set
	\draw[dashed,rounded corners=10]($(1) + (-.2,.25)$)rectangle($(2) +(.2,-.25)$);
	% specify mover at 2nd information set
	\node at ($(1)!.5!(2)$) {$P2$};
	\end{tikzpicture}
\end{figure}

	\end{enumerate}
\end{example} ~\

\begin{definition}[Information] ~\
	\begin{enumerate}
		\item \textbf{Information Set:} A player doesn't know which of the nodes in the information set she is actually at. Therefore, at any decision node in a player's information set, there must be the same possible actions.
		\item \textbf{Perfect Information:} A game is said to be of perfect information if each information set contains a single decision node. Otherwise, it is a game of \textbf{imperfect information}.
	\end{enumerate}
\end{definition}

\begin{definition}[Extensive Form Game] ~\
	A game in \textbf{extensive form} consists of:
	\begin{enumerate}[label=(\roman*\upshape)]
		\item A finite set of nodes $\mathcal{X}$, a finite set of possible actions $\mathcal{A}$, and a finite set of players $\{1, \dotsc, l\}$.
		\item A funktion $p \colon \mathcal{X} \rightarrow \{ \mathcal{X} \cup \emptyset \}$ specifying a single immediate predecessor of each node $x$; $p(x) \in \mathcal{X}$ expect for one element $x_{0}$, the \textbf{initial node}. The immediate \textbf{successor node} of $x$ are $s(x) = p^{-1}(x)$. ~\\
			To have a tree structure, a predecessor can never be a successor and vice versa. The set of \textbf{terminal nodes} $T = \{ x \in \mathcal{X} \colon s(x) = \emptyset \}$. All other nodes $X \setminus T$ are \textbf{decision nodes}.
		\item A function $\alpha \colon \mathcal{X} \setminus \{ x_{0} \} \rightarrow \mathcal{A}$ giving the action that leads to any non-initial node $x$ from its immediate predecessor $p(x)$ with $x', x'' \in s(x); x' \neq x'' \Rightarrow \alpha(x') \neq \alpha(x'')$. The set of choices at decision node $x$ is $c(x) = \{ a \in \mathcal{A} \colon a = \alpha(x') \text{ for some } x' \in s(x) \}$.
		\item A collection of information sets $\mathcal{H}$, and a function $H \colon \mathcal{X} \rightarrow \mathcal{H}$ assigning each decision node $x$ to an information set $H(x) \in \mathcal{H}$ with $c(x) = c(x')$ if H(x) = $H(x')$. ~\\
			The choices available at information set $H$ can be written as
			$$ C(H) = \{ a \in \mathcal{A} \colon a \in c(x) \text{ for } x \in H \}. $$
		\item A function $\iota \colon \mathcal{H} \rightarrow \{ 0, 1, \dotsc, l \}$ assigning a player to each information set ($i = 0$ 'nature'). ~\\
			The collection of player i's information set is denoted by
			$$ \mathcal{H}_i = \{ H \in \mathcal{H} \colon i = \iota(H) \}. $$
		\item A function $\rho \colon \mathcal{H}_0 \times \mathcal{A} \rightarrow [0,1]$ assigning a probability to each action of nature with $\rho(H,a) = 0$ if $a \notin C(H)$ und $\sum_{a \in C(H)} \rho(H, a) = 1$ for all $H \in \mathcal{H}_{0}$.
		\item A collection of payoff function $u = \{ u_1(\cdot), \dotsc, u_l(\cdot) \}$, where $u_i \colon T \rightarrow \R$.
	\end{enumerate}
	\textbf{A game in extensive form:} $\Gamma_E = \{ \mathcal{X}, \mathcal{A}, I, p(\cdot), \alpha(\cdot), \mathcal{H}, H(\cdot), \iota(\cdot), \rho(\cdot), u \}$.
\end{definition}

\begin{comment*}
	Restrictions of this definition:
	\begin{enumerate}
		\item Finite set of actions
		\item Finite number of moves
		\item Finite number 	of players
	\end{enumerate}
\end{comment*}

\begin{definition}[Strategy]
	Let $\mathcal{H}_i$ denote the collection of player $i$'s information sets, $\mathcal{A}$ the set of possible actions in the game, and $C(H) \subset \mathcal{A}$ the set of actions possible at information set $H$. A \textbf{strategy} for player $i$ is a function $s_i \colon \mathcal{H}_i \rightarrow \mathcal{A}$ such that $s_i(H) \in C(H)$ for all $H \in \mathcal{H}_i$.
\end{definition}

\begin{definition}[Normal Form Representation]
	For a game with $I$ players, the \textbf{normal form representation} $\Gamma_N$ specifies for each player $i$ a set of strategies $\mathcal{S}_{i}$ (with $s_i \in \mathcal{S}_i$) and a payoff function $u_i(s_1, \dotsc, s_l)$, formally 
	$$ \Gamma_N = [I, \{ S_i \}, \{ u_i(\cdot) \}]. $$
\end{definition} 

\newpage

\begin{definition} ~\
	\begin{enumerate}
		\item $s_i \colon \mathcal{H}_i \rightarrow \mathcal{A}$ describes deterministic choices at each $H \in \mathcal{H}_i$ and is called a \textbf{pure strategy}
		\item a \textbf{mixed strategy} is a probability distribution over all pure strategies $\sigma_i \colon \mathcal{S}_i \rightarrow [0, 1]$, with $\sigma_i(s_i) \geq 0$ and $\sum_{s_i \in \mathcal{S}_i} \sigma_i(s_i) = 1$.
		\item player $i$'s set of possible mixed strategies can be associated with the points of the simplex $\Delta(\mathcal{S}_i)$, called the \textbf{mixed extension} of $\mathcal{S}_i$.
		\item since we assume that individuals are expected utility maximisers, player $i$'s utility of a profile of mixed strategies $\sigma = \left( \sigma_i, \dotsc, \sigma_l \right)$ is given by
			$$ u_i(\sigma) = \sum_{s \in \mathcal{S}} [\sigma_1(s_1) \cdot \sigma_2(s_2) \cdot \dotsc \cdot \sigma_l(s_l)] \cdot u_i(s), $$
			where $s = (s_1, \dotsc, s_l)$.
	\end{enumerate}
\end{definition}

\begin{definition}[Behaviour Strategy]
	Given an extensive form game $\Gamma_E$, a \textbf{behaviour strategy} for player $i$ specifies for every information set $h \in \mathcal{H}_i$ and action $a \in C(H)$, a probability $\lambda_i(a, H) \geq 0$, with
	$$ \sum_{a \in C(H)} \lambda_i(a, H) = 1 \text{ for all } H \in \mathcal{H}_i. $$
\end{definition}

\begin{definition}[Perfect Recall]
	A player has \textbf{perfect recall} if he doesn't \enquote{forget} what she once knew, including her own actions.
\end{definition}

\begin{theorem}
	If $\Gamma_E$ is an extensive form game with perfect recall, then for any mixed strategy there is an outcome equivalent behaviour strategy and vice versa.	
\end{theorem}

\section{Rationalisable Strategies}

Central question of Game Theory: What should we expect to observe in a game played by rational players? Or more precisely: What should we expect to observe in a game played by rational players who are fully knowledgeable about the structure of the game and each others' rationality? ~\\

We first address the above question for simultaneous-move games, which we study using their normal form representation. We use the following notation:
\begin{itemize}
	\item $\Gamma_N = [I, \{ S_i \}, \{ u_i(\cdot)]$ if we consider pure strategies only, ~\\
		$\Gamma_{N} = [I, \{ \Delta(S_i)\}, \{ u_i(\cdot) \}]$ if we allow for mixed strategies
	\item $s_{-i} = (s_1, \dotsc, s_{i-1}, s_{i+1}, \dotsc, s_l) \in \mathcal{S}_{-i}$ where $\mathcal{S}_{-i} = S_1 \times \dotsc \times S_{i-1} \times S_{i+1} \times \cdots \times S_{l}$
	\item $s = (s_i, s_{-i})$
\end{itemize}

\begin{example}[Prisoners' Dilemma] ~\\
			\begin{figure}[h!] \centering
  				\begin{game}{2}{2}[Player $1$][Player $2$]
   	    			   	 	&	  don't confess    &  confess   \\
   	 				don't confess   &    $-2, -2$   & $-10, -1$  \\
   	 				confess   &    $-1, -10$   & $-5, -5$  \\
   				\end{game}
			\end{figure}
\end{example}

What should we expect to observe in the Prisoners' Dilemma?

\begin{definition}[Strictly Dominant Strategy]
	A strategy $s_i \in \mathcal{S}_i$ is strictly dominant for player $i$ in game $\Gamma_N = ]I, \{  \mathcal{S}_i \}, \{ u_i(\cdot)\}]$ if for all $s_i' \neq s_i$:
	$$ u_{i}(s_i, s_{-i}) > u_i(s_i', s_{-i})  $$
	for all $s_{i} \in \mathcal{S}_{-i}$.
\end{definition}
Applied to Prisoner's Dilemma: Confess is a strictly dominant strategy for each player.

\begin{definition}[Strictly Dominated Strategy]
	$s_i \in \mathcal{S}_i$ is \textbf{strictly dominated} for player $i$ in game $\Gamma_N$ if there exists another strategy $s_i' \in \mathcal{S}_i$ such that:
	$$ u_i(s_i', s_{-i}) \geq u_i(s_i, s_{-i}) $$
	for all $s_{-i} \in \mathcal{S}_{-i}$. In this case we say that $s_i'$ strictly dominates $s_i$.
\end{definition}


\begin{definition}[Weakly Dominated Strategy]
	$s_i \mathcal{S}_{i}$ is weakly dominated for player $i$ in game $\Gamma_N$ if there exists another strategy $s_i' \in \mathcal{S}_i$ such that:
	$$ u_i(s_i', s_{-i}) \geq u_i(s_i, s_{-i}) $$
	for all $s_{-i} \in \mathcal{S}_{-i}$, with strict inequality for at least one $s_{-i}$.
\end{definition}

\begin{example} ~\\
	\begin{figure}[h!] \centering
  				\begin{game}{3}{2}[Player $1$][Player $2$]
   	    			   	 	&	  L    &  R   \\
   	 				U   &    $1, -1$   & $-1, 1$  \\
   	 				M   &    $-1, 1$   & $1, -1$  \\
   					D   &    $-2, 5$   & $-3, 2$  \\
   				\end{game} $ \Rightarrow D \text{ is strictly dominated by } U \text{ and } M.$
   					\begin{game}{3}{2}[Player $1$][Player $2$]
   	    			   	 	&	  L    &  R   \\
   	 				U   &    $5, 1$   & $4, 0$  \\
   	 				M   &    $6, 0$   & $3, 1$  \\
   					D   &    $6,4$   & $4, 4$  \\
   				\end{game} $\Rightarrow U \text{ and } M \text{ are weakly dominated by } D.$
	\end{figure}
\end{example} ~\\


\begin{example}[Prisoners’ Dilemma – A Variation]
	Assume Prisoner 1 is the district attorney’s brother: If neither player confesses, player 1 is free	 ~\\
	\begin{figure}[h!] \centering
  				\begin{game}{2}{2}[Player $1$][Player $2$]
   	    			   	 	&	  don't confess    &  confess   \\
   	 				don't confess   &    $0, -2$   & $-10, -1$  \\
   	 				confess   &    $-1, -10$   & $-5, -5$  
   	   				\end{game} $$\Rightarrow D \text{ is strictly dominated by } U \text{ and } M.$$
	\end{figure} ~\
	
	$\Rightarrow \text{ Player 1 has no dominant strategy anymore}$.
\end{example}

In this game, the iterated elimination of strictly dominated strategies still leads to a unique prediction. In general, the order of elimination of strictly dominated strategies does not matter! How about iterated elimination of weakly dominated strategies?

\begin{definition}
	A strategy $\sigma_i \in \Delta(\mathcal{S}_i)$ is strictly dominated for $i$ in game $\Gamma_{N} = [I, \{ \Delta(\mathcal{S}_i)\}, \{ u_i(\cdot) \}]$ if there exists another strategy $\sigma_i' \in \Delta(\mathcal{S}_i)$ such that for all $\sigma_{-i} \in \Pi_{j \neq i} \Delta(\mathcal{S}_{j})$:
	$$ u_{i}(\sigma_i', \sigma_{-i}) > u_i(\sigma_i, \sigma_{-i}). $$
\end{definition}


\begin{proposition}
	Player $i$'s pure strategy $s_i \in \mathcal{S}_i$ is strictly dominated in a game $\Gamma_N = [I, \{ \Delta(\mathcal{S}_i)\}, \{ u_i(\cdot)\}]$ if and only if there exists another strategy $\sigma_i' \in \Delta(\mathcal{S}_i)$ such that
	$$ u_i(\sigma_i', s_{-i}) > u_i(s_i, s_{-i}) \text{ for all } s_{-i} \in \mathcal{S}_{-i}. $$
	
	\begin{proof}
		This follows because we can write
		$$ u_i(\sigma_i', \sigma_{-i}) - u_i(s_i, \sigma_{-i}) = \sum_{s_{-i} \in \mathcal{S}_{-i}} \left[ \Pi_{k \neq i} \sigma_{k}(s_{k}) \right] \left[ u_{i}(\sigma_i', s_{-i}) - u_{i}(s_i, s_{-i}) \right]. $$
		And this expression is positive for all $\sigma_{-i}$ if and only if $u_i(\sigma_i', s_{-i}) - u_{i}(s_{i}, s_{-i})$ is positive for all $s_{-i}$.
	\end{proof}
\end{proposition} ~\\

\begin{example} ~\
	\begin{figure}[h!] \centering
  				\begin{game}{3}{2}[Player $1$][Player $2$]
   	    			   	 	&	  L    &  R   \\
   	 				U   &    $10, 1$   & $0, 4$  \\
   	 				M   &    $4, 2$   & $4, 3$ \\
   	 				D   &    $0, 5$   & $10, 2$  

   	   				\end{game} $$\Rightarrow \frac{1}{2}U+ \frac{1}{2} D \text{ strictly dominates } M.$$
	\end{figure}
\end{example}

\begin{definition}[Best response]
	The strategy $\sigma_i$ is a \textbf{best response} for player $i$ to her rivals' strategies $\sigma_{-i}$ if:
	$$ u_i(\sigma_i, \sigma_{-i}) \geq u_i(\sigma_i', \sigma_{-i}) $$
	for all $\sigma_i' \in \Delta(\mathcal{S}_i)$. Strategy $\sigma_i$ is never a best response if there is no $\sigma_{-i}$ for which $\sigma_{i}$ is a best response.
\end{definition}

\begin{definition}[Rationalisable Strategies]
	In game $\Gamma_N = [I, \{ \Delta(\mathcal{S}_i) \}, \{ u_i(\cdot) \}]$, the strategies in $\Delta(\mathcal{S}_i)$ that survive the iterated elimination of strategies that are never a best response are known as player $i$'s \textbf{rationalisable strategies}.
\end{definition} ~\newpage

\begin{example} ~\\
		\begin{figure}[h!] \centering
  				\begin{game}{4}{4}[Player $1$][Player $2$]
   	    			   	 	& $b_1$ & $b_2$ & $b_3$ & $b_4$   \\
   	 				$a_1$   &    $0, \underline{7}$   & $2, 5$&    $\underline{7}, 0$   & $0, 1$  \\
   	 				$a_2$   &    $5, 2$   & $\underline{3}, \underline{3}$&    $5, 2$   & $0, 1$ \\
   	 				$a_3$   &    $\underline{7}, 0$   & $2, 5$ &    $0, \underline{7}$   & $0, 1$  \\
					$a_4$   &    $0, \underline{0}$   & $0, -2$ &    $0, \underline{0}$   & $\underline{10}, -1$ 
   	   				\end{game} $$\Rightarrow \frac{1}{2}U+ \frac{1}{2} D \text{ strictly dominates } M.$$
	\end{figure} ~\\
	$\Rightarrow b_4$ is never best response for player 2 and \textit{then} $a_4$ is never best response for player 1. ~\\
	$\Rightarrow \{a_1, a_2, a_3\}$ and $\{ b_1, b_2, b_3 \}$ are the rationalisable strategies in this game.
\end{example}

\newpage


\section{Nash Equilibrium}

\section{Subgame Perfection in Dynamic Games}

\section{Excercises}

\chapter{Kooperative Spiele}

\section{Der Kern}

\section{Der Shapley-Wert}

\section{Einfache Spiele}

\section{Konvexe Spiele}

\section{Übungen}

\chapter{Evolutionäre Spieltheorie}

\newpage

% todo Bookmark Vorlesung / Übung
\phantomsection \appendix \pagenumbering{Roman}  \cftaddtitleline{toc}{chapter}{Appendix}{}
\cftaddtitleline{toc}{section}{Übungen}{I}




\printindex

\end{document}