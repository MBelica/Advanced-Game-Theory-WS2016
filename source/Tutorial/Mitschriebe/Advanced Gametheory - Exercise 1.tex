\documentclass[12pt]{extreport} % Schriftgröße: 8pt, 9pt, 10pt, 11pt, 12pt, 14pt, 17pt oder 20pt

%% Packages
\usepackage{scrextend}
\usepackage{amssymb}
\usepackage{amsthm}
\usepackage{chngcntr}
\usepackage{cmap}
\usepackage{color}
\usepackage{hyperref}
\usepackage{lmodern}
\usepackage{makeidx}
\usepackage{mathtools} 
\usepackage{xpatch}
\usepackage{pgfplots}
\pgfplotsset{compat=1.7}
\usetikzlibrary{calc}	
\usetikzlibrary{matrix}	
\usepackage{sgame, tikz} % Game theory packages
\usetikzlibrary{trees, calc} % For extensive form games
\usepackage{subfig} % Manipulation and reference of small or sub figures and tables


\usepackage{amsmath}
\usepackage[inline]{enumitem}

\usepackage{cmap}

\usepackage{microtype}
\usepackage{sgame, tikz} % Game theory packages
\usetikzlibrary{trees, calc} % For extensive form games
\usepackage{subfig} % Manipulation and reference of small or sub figures and tables


% Language Setup (Deutsch)
\usepackage[utf8]{inputenc} 
\usepackage[T1]{fontenc} 
\usepackage[english]{babel}

% Options
\makeatletter%%  
  % Linkfarbe, {0,0.35,0.35} für Türkis, {0,0,0} für Schwarz 
  \definecolor{linkcolor}{rgb}{0,0.35,0.35}
  % Zeilenabstand für bessere Leserlichkeit
  \def\mystretch{1.2} 
  % Publisher definieren
  \newcommand\publishers[1]{\newcommand\@publishers{#1}} 
  % Enumerate im 1. Level: \alph für a), b), ...
  \renewcommand{\labelenumi}{\alph{enumi})} 
  % Enumerate im 2. Level: \roman für (i), (ii), ...
  \renewcommand{\labelenumii}{(\roman{enumii})}
  % Zeileneinrückung am Anfang des Absatzes
  \setlength{\parindent}{0pt} 
  % Verweise auf Enumerate, z.B.: 3.2 a)
  \setlist[enumerate,1]{ref={\thesatz ~ \alph*)}}
  % Für das Proof-Environment: 'Beweis:' anstatt 'Beweis.'
  \xpatchcmd{\proof}{\@addpunct{.}}{\@addpunct{:}}{}{} 
  % Nummerierung der Bilder, z.B.: Abbildung 4.1
  \@ifundefined{thechapter}{}{\def\thefigure{\thechapter.\arabic{figure}}} 
\makeatother%

% Meta Setup (Für Titelblatt und Metadaten im PDF)
\title{Advanced}
\author{G. Herzog, C. Schmoeger}
\date{Wintersemester 2016/17}
\publishers{Karlsruher Institut für Technologie}


%% Math. Definitions
\newcommand{\C}{\mathbb{C}}
\newcommand{\N}{\mathbb{N}}
\newcommand{\Q}{\mathbb{Q}}
\newcommand{\R}{\mathbb{R}}
\newcommand{\Z}{\mathbb{Z}}

%% Theorems (unnamedtheorem = Theorem ohne Namen)
\newtheoremstyle{named}{}{}{\normalfont}{}{\bfseries}{:}{0.25em}{#2 \thmnote{#3}}
\newtheoremstyle{itshape}{}{}{\itshape}{}{\bfseries}{:}{ }{}
\newtheoremstyle{normal}{}{}{\normalfont}{}{\bfseries}{:}{ }{}
\renewcommand*{\qed}{\hfill\ensuremath{\square}}

\theoremstyle{named}
\newtheorem{unnamedtheorem}{Theorem} \counterwithin{unnamedtheorem}{chapter}

\theoremstyle{itshape}
\newtheorem{satz}[unnamedtheorem]{Satz}	
\newtheorem*{definition}{Definition}

\theoremstyle{normal}
\newtheorem{beispiel}[unnamedtheorem]{Beispiel}
\newtheorem{folgerung}[unnamedtheorem]{Folgerung}
\newtheorem{hilfssatz}[unnamedtheorem]{Hilfssatz}
\newtheorem{anwendung}[unnamedtheorem]{Anwendung}
\newtheorem{anwendungen}[unnamedtheorem]{Anwendungen}
\newtheorem*{beispiel*}{Beispiel}
\newtheorem*{beispiele}{Beispiele}
\newtheorem*{bemerkung}{Bemerkung} 
\newtheorem*{bemerkungen}{Bemerkungen}
\newtheorem*{bezeichnung}{Bezeichnung}
\newtheorem*{eigenschaften}{Eigenschaften}
\newtheorem*{folgerung*}{Folgerung}
\newtheorem*{folgerungen}{Folgerungen}
\newtheorem*{hilfssatz*}{Hilfssatz}
\newtheorem*{regeln}{Regeln}
\newtheorem*{schreibweise}{Schreibweise}
\newtheorem*{schreibweisen}{Schreibweisen}
\newtheorem*{uebung}{Übung}
\newtheorem*{vereinbarung}{Vereinbarung}

%% Template
\makeatletter%
\DeclareUnicodeCharacter{00A0}{ } \pgfplotsset{compat=1.7} \hypersetup{colorlinks,breaklinks, urlcolor=linkcolor, linkcolor=linkcolor, pdftitle=\@title, pdfauthor=\@author, pdfsubject=\@title, pdfcreator=\@publishers}\DeclareOption*{\PassOptionsToClass{\CurrentOption}{report}} \ProcessOptions \def\baselinestretch{\mystretch} \setlength{\oddsidemargin}{0.125in} \setlength{\evensidemargin}{0.125in} \setlength{\topmargin}{0.5in} \setlength{\textwidth}{6.25in} \setlength{\textheight}{8in} \addtolength{\topmargin}{-\headheight} \addtolength{\topmargin}{-\headsep} \def\pulldownheader{ \addtolength{\topmargin}{\headheight} \addtolength{\topmargin}{\headsep} \addtolength{\textheight}{-\headheight} \addtolength{\textheight}{-\headsep} } \def\pullupfooter{ \addtolength{\textheight}{-\footskip} } \def\ps@headings{\let\@mkboth\markboth \def\@oddfoot{} \def\@evenfoot{} \def\@oddhead{\hbox {}\sl \rightmark \hfil \rm\thepage} \def\chaptermark##1{\markright {\uppercase{\ifnum \c@secnumdepth >\m@ne \@chapapp\ \thechapter. \ \fi ##1}}} \pulldownheader } \def\ps@myheadings{\let\@mkboth\@gobbletwo \def\@oddfoot{} \def\@evenfoot{} \def\sectionmark##1{} \def\subsectionmark##1{}  \def\@evenhead{\rm \thepage\hfil\sl\leftmark\hbox {}} \def\@oddhead{\hbox{}\sl\rightmark \hfil \rm\thepage} \pulldownheader }	\def\chapter{\cleardoublepage  \thispagestyle{plain} \global\@topnum\z@ \@afterindentfalse \secdef\@chapter\@schapter} \def\@makeschapterhead#1{ {\parindent \z@ \raggedright \normalfont \interlinepenalty\@M \Huge \bfseries  #1\par\nobreak \vskip 40\p@ }} \newcommand{\indexsection}{chapter} \patchcmd{\@makechapterhead}{\vspace*{50\p@}}{}{}{}
	% Titlepage
	\def\maketitle{ \begin{titlepage} 
			~\vspace{3cm} 
		\begin{center} {\Huge \@title} \end{center} 
	 		\vspace*{1cm} 
	 	\begin{center} {\large \@author} \end{center} 
	 	\begin{center} \@date \end{center} 
	 		\vspace*{7cm} 
	 	\begin{center} \@publishers \end{center} 
	 		\vfill 
	\end{titlepage} }
\makeatother%

% Indexdatei erstellen
\makeindex 

\begin{document}


\section*{Advanced Game Theory - Exercise 1}

\subsection*{Aufgabe 1.1}
	Es gilt $|S| = \Pi_{n=1}^{N} M_{n}$: \quad $S = M_{1} \times \dots \times M$.

\subsection*{Aufgabe 1.2}	


	% Node styles
	\tikzset{
	% Two node styles for game trees: solid and hollow
	solid node/.style={circle,draw,inner sep=1.5,fill=black},
	hollow node/.style={circle,draw,inner sep=1.5}
	}

\begin{figure}[htbp]
	\centering
		\caption*{Extensive form game with imperfect information}
	\begin{tikzpicture}[scale=1.5,font=\footnotesize]
	% Specify spacing for each level of the tree
	\tikzstyle{level 1}=[level distance=15mm,sibling distance=40mm]
	\tikzstyle{level 2}=[level distance=15mm,sibling distance=22mm]
	\tikzstyle{level 3}=[level distance=15mm,sibling distance=12.5mm]
	% The Tree
	\node(0)[hollow node,label=above:{$P1$}]{}
	child{node(1)[solid node]{}
		node[label=below:{$T_{0}$}]{} 
		edge from parent node[left,xshift=-6]{$L$}
	}	
	child{node(2)[solid node]{}
		child{node(4)[solid node,label=right:{}]{} 
			child{node(8)[label=below:{$T_{1}$}]{} edge from parent node[left]{$x$}}
			child{node(9)[label=below:{$T_{2}$}]{} edge from parent node[right]{$y$}}
			edge from parent node[right]{$l$}}
		child{node(5)[solid node,label=right:{}]{} 
			child{node(10)[label=below:{$T_{3}$}]{} edge from parent node[left]{$x$}}
			child{node(11)[label=below:{$T_{4}$}]{} edge from parent node[right]{$y$}}
			edge from parent node[right]{$r$}}
		edge from parent node[left,xshift=-3]{$M$}
	}
	child{node(3)[solid node]{}
		child{node(6)[solid node,label=right:{}]{} 
			child{node(12)[label=below:{$T_{5}$}]{} edge from parent node[left]{$x$}}
			child{node(13)[label=below:{$T_{6}$}]{} edge from parent node[right]{$y$}}
			edge from parent node[right]{$l$}}
		child{node(7)[solid node,label=right:{}]{} 
			child{node(14)[label=below:{$T_{7}$}]{} edge from parent node[left]{$x$}}
			child{node(15)[label=below:{$T_{8}$}]{} edge from parent node[right]{$y$}}
			edge from parent node[right]{$r$}}
		edge from parent node[right,xshift=6]{$R$}
	};
	% information sets
	\draw[dashed,rounded corners=10]($(2) + (-.2,.25)$)rectangle($(3) +(.2,-.25)$);
	\draw[dashed,rounded corners=10]($(4) + (-.2,.25)$)rectangle($(5) +(.2,-.25)$);
	\draw[dashed,rounded corners=10]($(6) + (-.2,.25)$)rectangle($(7) +(.2,-.25)$);
	% specify mover at 2nd information set
	\node at ($(2)!.5!(3)$) {$P2$};
	% specify mover at 3nd information set
	\node at ($(4)!.5!(5)$) {$P1$};
	% specify mover at 4nd information set
	\node at ($(6)!.5!(7)$) {$P1$};
  \end{tikzpicture}
\end{figure}


\begin{enumerate}
	\item Die Strategieräume sind:
	  \begin{align*}
 		S_{1} & = \big\{ (L, x, x), (L, x, y), (L, y, x), (L, y, y), ~\hspace{7cm} \\
 			  & ~\qquad (M, x, x), (M, x, y), (M, y, x), (M, y, y), \\
 			  & ~\qquad (R, x, x), (R, x, y), (R, y, x), (R, y, y) \big\} \\
 			  & = \left\{ S_{1}^{1}, \dotsc, S_{1}^{12} \right\} \\
 		S_{2} & = \left\{ (l), (r) \right\}
 	  \end{align*}
	\item It must hold that:
		\begin{align*}
			p_{1} + p_{2} + p_{3} & = 1 \\
			q_{1} + q_{2} & = 1 \\
			r_{1} + r_{2} & = 1
		\end{align*}
		Example of a behaviour strategy: $(p_{1}L + p_{2}M + p_{3}R, q_{1} x + q_{2}y, r_{1}x + r_{2} y)$ \\
		Example of a mixed strategy: $\sum_{i=1}^{12} p_{i} S_{1}^{i}$ \\
		For player 2 there is nothing to show. \\ 
		
		Probability distribution of the outcomes:
		$$ p_{1}, ~ p_{2} \sigma(l) q_{1}, ~ p_{2} \sigma(l) q_{2}, ~ p_{2} \sigma(r) q_{1}, ~ p_{2} \sigma(r) q_{2}, \dotsc $$
		The following mixed strategy of player 1 is realisation equivalent
		$$ \left( p_{1} S^{1} + p_{2} q_{1} S_{1}^{5} + p_{2} q_{2} S_{1}^{7} + p_{3} r_{1} S_{1}^{9} + p_{3} r_{2} S_{1}^{10} \right) $$
		z.z.: ~ $ 1 = p_{1} + p_{2} q_{1} + p_{2} q_{2} + p_{3} r_{1} + p_{3} r_{3}$, \quad klar.
\end{enumerate}

\subsection*{Aufgabe 1.3}
\begin{table}[!htbp]
\centering
	
\begin{game}{2}{4}[Player 1][Player 2]
	    &  LL     &  L & M, & R    \\
	 U  &  $100, 2$ & $-100, 1$ & $0,0$ & $-100, -100$  \\
	 D  &  $-100, -100$ & $100, -49$ & $1, 0$ & $100, 2$ \\
\end{game}
\end{table}

\end{document}